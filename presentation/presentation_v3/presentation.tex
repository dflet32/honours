\documentclass{beamer}
\usetheme{metropolis}

%\usepackage{microtype}
\usepackage{changepage}
\usepackage{graphicx}
\usepackage{wrapfig}
\usepackage{tikz}
\usetikzlibrary{calc}  % Add calc library for better TikZ control
\usepackage{subcaption} % For side-by-side figures with captions


\theoremstyle{definition}
\newtheorem{noitalicstheorem}{Theorem}

\DeclareMathOperator{\GL}{GL}
\DeclareMathOperator{\M}{M}

\newcommand{\C}{k}
\renewcommand{\O}{\mathcal{O}}
\newcommand{\Q}{\mathbb{Q}}
\newcommand{\Z}{\mathbb{Z}}
\newcommand{\p}{\mathfrak{p}}

\usetheme{metropolis}
\definecolor{mynavy}{RGB}{35, 55, 59}
\definecolor{myorange}{RGB}{235, 129, 27}
\definecolor{myteal}{RGB}{92, 188, 172}
\definecolor{mygreen}{RGB}{11, 176, 161}
\definecolor{myblue}{RGB}{130, 158, 117}
\metroset{block=fill}

%\setbeamercolor{block body}{bg=transparent}
%\setbeamercolor{block title}{bg=transparent}

\theoremstyle{definition}
\newtheorem{singularitiestheorem}{Theorem (Cones detect singularities)}
\newtheorem{proposition}{Proposition}


\title{Toric Varieties}
%\subtitle{Three perspectives}
\author{Declan Fletcher}
\date{October 2024}

\begin{document}
\begin{frame}
\titlepage
\end{frame}

\begin{frame}
\frametitle{Algebraic varieties}
\centering
\begin{minipage}[t]{0.49\textwidth}
\centering
\pause
\begin{figure}
\begin{tikzpicture}
  \draw[ultra thick, myorange, samples=100, smooth] 
    plot[domain=0:360] ({cos(\x)}, {sin(\x)});
\end{tikzpicture}
\vspace{0.05cm}
\caption*{$x^2+y^2=1$}
\end{figure}
\end{minipage}
\hfill
\begin{minipage}[t]{0.49\textwidth}
    \centering
\vspace{-1cm}
\pause
\begin{figure}
    \includegraphics[width=0.75\linewidth]{orange_sphere}
	\vspace{-1.2cm}
     \caption*{$x^2+y^2+z^2=1$}
\end{figure}
\end{minipage}
%\vfill
\begin{minipage}[t]{0.49\textwidth}
    \centering
\vspace{-0cm}
\pause
\begin{figure}
\begin{tikzpicture}
  \draw[ultra thick, myorange, samples=100, smooth, domain=-2:0.95, variable=\x] 
    plot ({\x}, {sqrt((\x)^3 + 2*(\x)^2)});
  \draw[ultra thick, myorange, samples=100, smooth, domain=-2:0.95, variable=\x] 
    plot ({\x}, {-sqrt((\x)^3 + 2*(\x)^2)});
\end{tikzpicture}
\caption*{$y^2=x^3+2x^2$}
\end{figure}
\end{minipage}
\hfill
\begin{minipage}[t]{0.49\textwidth}
    \centering
\vspace{-0.2cm}
\pause
\begin{figure}
    \includegraphics[width=0.7\linewidth]{orange_cone}
	\vspace{-.25cm}
    \caption*{$xy=z^2$}
\end{figure}
\end{minipage}
\end{frame}

\begin{frame}
\frametitle{What are toric varieties, and why study them?}
%\pause
%In general, varieties these are complicated and difficult to study.

\pause
\alert{Toric varieties} are a special class of varieties determined by elementary \alert{convex} geometry.

\pause
They serve as testing grounds for conjectures and theorems in algebraic geometry.

\pause
For example, the \alert{Hodge conjecture}---a Millennium Prize Problem---is proven for toric varieties but unresolved in general.

\pause
\textbf{Goal of the talk}: introduce toric varieties and explain some of their properties.

\end{frame}

\begin{frame}[t]
\frametitle{The definition of a variety}
\onslide<1->{
Varieties are sets of solutions $(a_1, \ldots, a_n) \in \mathbb{C}^n$ to poly.\ equations
$$f_1(a_1, \ldots, a_n) = 0, \quad \ldots, \quad f_s(a_1, \ldots, a_n) = 0.$$
}

\vspace{-0.2cm}

\begin{minipage}{0.45\textwidth}
\linespread{0.2}\selectfont  % Adjust the line spacing inside the minipage
\onslide<2->{Choose the zero polynomial:
$$\rightsquigarrow \mathbb{C}^n.$$}
\onslide<3->{\noindent Choose $y^2 = x^3 + 2x^2$:
$$\rightsquigarrow \text{\textcolor{blue}{singular curve}.}$$}
\onslide<4->{\noindent Choose $xy = 1$:
$$\rightsquigarrow \{(t, t^{-1}) : t \in \mathbb{C}^\times\} \cong \mathbb{C}^\times.$$}
\end{minipage}
\hfill
\begin{minipage}{0.45\textwidth}
\onslide<3->{
\begin{tikzpicture}
  % Draw axes
  \draw[->] (-2.7, 0) -- (2.2, 0); % node[right] {$x$};
  \draw[->] (0, -2) -- (0, 2); % node[above] {$y$};

  % Plot the curve y^2 = x^3 - x + 1 manually
  \draw[thick, blue, samples=100, smooth, domain=-2:1.1, variable=\x] 
    plot ({\x}, {sqrt((\x)^3 + 2*(\x)^2)});
  \draw[thick, blue, samples=100, smooth, domain=-2:1.1, variable=\x] 
    plot ({\x}, {-sqrt((\x)^3 + 2*(\x)^2)});
\end{tikzpicture}
}
\end{minipage}

\onslide<5->{
$\mathbb{C}^\times$ is called the algebraic \alert{torus}. The $d$-dimensional torus is $(\mathbb{C}^\times)^d$.
}
\end{frame}

\begin{frame}
\frametitle{Convex cones}
To study toric varieties, we need to understand convex cones.

\onslide<2->{We consider \alert{polyhedral cones}. These are sets in $\mathbb{R}^n$ of the form
$$\sigma = \mathrm{span}_{\mathbb{R}_{\ge 0}}\{v_1, \ldots, v_r\},$$
for some $v_1, \ldots, v_r \in \mathbb{R}^n$.}

\onslide<4->{We call $\sigma$ \alert{rational} if we can take each $v_i \in \mathbb{Z}^n$.}

\pause
\begin{minipage}[t]{0.49\textwidth}
\centering
\onslide<3->{
\begin{figure}
\begin{tikzpicture}
            \fill[gray!30] (0,0) -- (-1.1,1.1) -- (-1.1,1.6) -- (1.1,1.6) -- (1.1, 1.1) -- cycle;
            
            % Axes
            \draw[->] (-1.25,0) -- (1.25,0);
            \draw[->] (0,-0.75) -- (0,1.75);
            
            % Lattice points
            \foreach \x in {-1,-0.5,0,0.5,1,}
                \foreach \y in {-0.5,0,0.5,1,1.5}
                    \fill (\x,\y) circle (1pt);

            % Vectors
            \draw[->, ultra thick] (0,0) -- (-0.5,0.5);
            \draw[->, ultra thick] (0,0) -- (0.5,0.5);
            \node at (0.75, 1.25) {$\sigma$};
\end{tikzpicture}
\end{figure}}
\end{minipage}
\hfill
\begin{minipage}[t]{0.49\textwidth}
    \centering
\onslide<3->{\begin{figure}
\begin{tikzpicture}
            \fill[gray!30] (0,0) -- (-0.8,1.6) -- (1.1,1.6) -- (1.1,0) -- cycle;
            
            % Axes
            \draw[->] (-1.25,0) -- (1.25,0);
            \draw[->] (0,-0.75) -- (0,1.75);
            
            % Lattice points
            \foreach \x in {-1,-0.5,0,0.5,1,}
                \foreach \y in {-0.5,0,0.5,1,1.5}
                    \fill (\x,\y) circle (1pt);

            % Vectors
            \draw[->, ultra thick] (0,0) -- (0.5,0);
            \draw[->, ultra thick] (0,0) -- (-0.5,1);
            \node at (0.75, 1.25) {$\sigma$};
\end{tikzpicture}
\end{figure}}
\end{minipage}

\end{frame}


\begin{frame}
\frametitle{Dual cones}
\textbf{Fix}: cone $\sigma$.
\pause
The \alert{dual cone} is the set of linear functionals which are non-negative on $\sigma$:
$$\sigma^\vee := \{u \in (\mathbb{R}^n)^* : u(v) \ge 0 \text{ for all } v \in \sigma\}.$$
\begin{figure}[H]
    \centering
    \pause \begin{tikzpicture}[scale=1.2]
        % Left diagram
        \begin{scope}[shift={(-2,0)}]
            % Shaded area
            \fill[gray!30] (0,0) -- (-1.05, 2.1) -- (2.1,2.1) -- (2.1,0) -- cycle;
            
            % Axes
            \draw[->] (-1.25,0) -- (2.25,0);
            \draw[->] (0,-0.25) -- (0,2.25);
            
            % Lattice points
            \foreach \x in {-1,-0.5,0,0.5,1,1.5,2}
                \foreach \y in {0,0.5,1,1.5,2}
                    \fill (\x,\y) circle (1pt);

            % Vectors
            \draw[->, ultra thick] (0,0) -- (0.5,0);
            \draw[->, ultra thick] (0,0) -- (-0.5,1);
            \node at (1.75, 1.75) {$\sigma$};
        \end{scope}

        % Right diagram
        \begin{scope}[shift={(2,0)}]
            % Shaded area
            \fill[gray!30] (0,0) -- (0,2.1) -- (2.1,2.1) -- (2.1, 1.05) -- cycle;
            
            % Axes
            \draw[->] (-1.25,0) -- (2.25,0);
            \draw[->] (0,-0.25) -- (0,2.25);
            
            % Lattice points
            \foreach \x in {-1,-0.5,0,0.5,1,1.5,2}
                \foreach \y in {0,0.5,1,1.5,2} {
                    \fill (\x,\y) circle (1pt);
                }
                
%			  % Labels
%			  \node[anchor=south west] at (0, 0) {\!\!\! \tiny \alert{$1$}};
%			  
%			  %\node[anchor=south] at (-0.5, 0) {\tiny \alert{$X^{-1}$}};
%			  %\node[anchor=south] at (-0.5, 0.5) {\tiny \alert{$X^{-1} Y$}};
%			  %\node[anchor=south] at (-0.5, 1) {\tiny \alert{$X^{-1} Y^2$}};
%			  %\node[anchor=south] at (-0.5, 1.5) {\tiny \alert{$X^{-1} Y^3$}};
%			  
%			  \node[anchor=south west] at (0, 0.5) {\!\!\! \tiny \alert{$Y$}};
%			  \node[anchor=south west] at (0, 1) {\!\!\! \tiny \alert{$Y^2$}};
%			  %\node[anchor=south] at (0, 1.5) {\tiny \alert{$Y^3$}};
%			  \node[anchor=south west] at (0.5, 0) {\!\!\!\! \tiny \alert{$X$}};
%			  \node[anchor=south west] at (0.5, 0.5) {\!\!\!\! \tiny \alert{$X Y$}};
%			  \node[anchor=south west] at (0.5, 1) {\!\!\!\!  \tiny \alert{$X Y^2$}};
%			  %\node[anchor=south] at (0.5, 1.5) {\tiny \alert{$X Y^3$}};
%			  \node[anchor=south west] at (1, 0) {\!\!\!\! \tiny \alert{$X^2$}};
%			  \node[anchor=south west] at (1, 0.5) {\!\!\!\! \tiny \alert{$X^2 Y$}};
%			  \node[anchor=south west] at (1, 1) {\!\!\!\! \tiny \alert{$X^2 Y^2$}};
%			  %\node[anchor=south] at (1, 1.5) {\tiny \alert{$X^2 Y^3$}};

            % Vectors
            \draw[->, ultra thick] (0,0) -- (0,0.5);
            \draw[->, ultra thick] (0,0) -- (1,0.5);
            \node at (1.75, 1.75) {\small{$\sigma^\vee$}};
        \end{scope}
    \end{tikzpicture}
\end{figure}
\end{frame}

\begin{frame}
\frametitle{An example of a toric variety}
\textbf{Fix}: cone $\sigma$, dual $\sigma^\vee$. \pause Monomials live on integer points in $(\mathbb{R}^n)^*$:
% \centerline{\includegraphics[width=0.8\textwidth]{cone_and_dual_with_monomials}}
\vspace{-0.2cm}
\begin{figure}[H]
    \centering
    \begin{tikzpicture}[scale=1.2]
        % Left diagram
        \begin{scope}[shift={(-2,0)}]
            % Shaded area
            \fill[gray!30] (0,0) -- (-1.05, 2.1) -- (2.1,2.1) -- (2.1,0) -- cycle;
            
            % Axes
            \draw[->] (-1.25,0) -- (2.25,0);
            \draw[->] (0,-0.25) -- (0,2.25);
            
            % Lattice points
            \foreach \x in {-1,-0.5,0,0.5,1,1.5,2}
                \foreach \y in {0,0.5,1,1.5,2}
                    \fill (\x,\y) circle (1pt);

            % Vectors
            %\draw[->, ultra thick] (0,0) -- (0.5,0);
            %\draw[->, ultra thick] (0,0) -- (-0.5,1);
            \node at (1.75, 1.75) {$\sigma$};
        \end{scope}

        % Right diagram
        \begin{scope}[shift={(2,0)}]
            % Shaded area
            \fill[gray!30] (0,0) -- (0,2.1) -- (2.1,2.1) -- (2.1, 1.05) -- cycle;
            
            % Axes
            \draw[->] (-1.25,0) -- (2.25,0);
            \draw[->] (0,-0.25) -- (0,2.25);
            
            % Lattice points
            \foreach \x in {-1,-0.5,0,0.5,1,1.5,2}
                \foreach \y in {0,0.5,1,1.5,2} {
                    \fill (\x,\y) circle (1pt);
                }
                
			  % Labels
			  \node[anchor=south west] at (0, 0) {\!\!\! \scriptsize \alert{$1$}};
			  
			  %\node[anchor=south] at (-0.5, 0) {\tiny \alert{$X^{-1}$}};
			  %\node[anchor=south] at (-0.5, 0.5) {\tiny \alert{$X^{-1} Y$}};
			  %\node[anchor=south] at (-0.5, 1) {\tiny \alert{$X^{-1} Y^2$}};
			  %\node[anchor=south] at (-0.5, 1.5) {\tiny \alert{$X^{-1} Y^3$}};
			
			  %\node[anchor=south ] at (-1, 0) {\, \scriptsize \alert{$r^{-2}$}};
			  %\node[anchor=south ] at (-1, 0.5) { \scriptsize \alert{$r^{-2} s$}};
			  %\node[anchor=south ] at (-1, 1) { \scriptsize \alert{$r^{-2} s^2$}};
  
			  \node[anchor=south ] at (-0.5, 0) {\scriptsize \alert{$r^{-1}$}};
			  \node[anchor=south ] at (-0.5, 0.5) { \scriptsize \alert{$r^{-1} s$}};
			  \node[anchor=south ] at (-0.5, 1) {\scriptsize \alert{$r^{-1} s^2$}};

			  \node[anchor=south west] at (0, 0.5) {\!\!\! \scriptsize \alert{$s$}};
			  \node[anchor=south west] at (0, 1) {\!\!\! \scriptsize \alert{$s^2$}};
			  %\node[anchor=south] at (0, 1.5) {\tiny \alert{$Y^3$}};
			  \node[anchor=south west] at (0.5, 0) {\!\!\!\! \scriptsize \alert{$r$}};
			  \node[anchor=south west] at (0.5, 0.5) {\!\!\!\! \scriptsize \alert{$r s$}};
			  \node[anchor=south west] at (0.5, 1) {\!\!\!\!  \scriptsize \alert{$r s^2$}};
			  %\node[anchor=south] at (0.5, 1.5) {\tiny \alert{$X Y^3$}};
			  \node[anchor=south west] at (1, 0) {\!\!\!\! \scriptsize \alert{$r^2$}};
			  \node[anchor=south west] at (1, 0.5) {\!\!\!\! \scriptsize \alert{$r^2 s$}};
			  \node[anchor=south west] at (1, 1) {\!\!\!\! \scriptsize \alert{$r^2 s^2$}};
			  %\node[anchor=south] at (1, 1.5) {\tiny \alert{$X^2 Y^3$}};

            % Vectors
            %\draw[->, ultra thick] (0,0) -- (0,0.5);
            %\draw[->, ultra thick] (0,0) -- (1,0.5);
            \node at (1.75, 1.75) {\small{$\sigma^\vee$}};
        \end{scope}
    \end{tikzpicture}
\end{figure}

\vspace{-0.3cm}

\pause
We create a ring using the monomials in $\sigma^\vee$:
\begin{align*}
	\mathbb{C}[1, s, r^2 s, r s, s^2, rs^2, \ldots] &= \mathbb{C}[s, r^2 s, r s] \\
	&\cong \mathbb{C}[x, y, z]/(xy - z^2).
\end{align*}

\pause
The toric variety $U_\sigma$ is the set of solutions to $xy - z^2 = 0$ in $\mathbb{C}^3$:
$$xy = z^2.$$
\end{frame}

\begin{frame}
\frametitle{The definition of a toric variety}
\textbf{Fix}: cone $\sigma$, dual $\sigma^\vee$. 
The previous construction generalises.

\pause
We associate monomials $x_1^{i_1} \cdots x_n^{i_n}$ to integer points in $(\mathbb{R}^n)^*$.

\pause
We create a ring using the monomials lying in $\sigma^\vee$, called $R_\sigma$.

\pause
$R_\sigma$ is finitely generated, so it's given by generators and relations:
$$R_\sigma = \mathbb{C}[y_1, \ldots, y_m]/(f_1, \ldots, f_s).$$

\pause
The \alert{toric variety} $U_\sigma$ is the subset of $\mathbb{C}^m$ defined by the equations
$$f_1(a_1, \ldots, a_m) = 0, \quad \ldots, \quad  f_s(a_1, \ldots, a_m) = 0.$$
\end{frame}


\begin{frame}
\frametitle{Cones detect singularities}

\onslide<2->{
\begin{theorem}
The toric variety $U_\sigma$ is non-singular if and only if $\sigma$ is generated by a subset of a basis for $\mathbb{Z}^n$.
\end{theorem}}


%\centerline{\includegraphics[width=\textwidth]{cone_and_variety}}
\onslide<3->{\begin{minipage}[t]{0.49\textwidth}
\centering
\begin{figure}
    \includegraphics[width=0.7\linewidth]{orange_cone}
	\vspace{0cm}
    \caption*{$xy=z^2$}
\end{figure}
\end{minipage}}
\hfill
\onslide<3->{\begin{minipage}[t]{0.49\textwidth}
\centering
\vspace{0.5cm}
\begin{figure}
    \begin{tikzpicture}[scale=1.2]
        % Left diagram

            % Shaded area
            \fill[gray!30] (0,0) -- (-1.05, 2.1) -- (2.1,2.1) -- (2.1,0) -- cycle;
            
            % Axes
            \draw[->] (-1.25,0) -- (2.25,0);
            \draw[->] (0,-0.25) -- (0,2.25);
            
            % Lattice points
            \foreach \x in {-1,-0.5,0,0.5,1,1.5,2}
                \foreach \y in {0,0.5,1,1.5,2}
                    \fill (\x,\y) circle (1pt);

            % Vectors
            \onslide<3>{\draw[->, ultra thick] (0,0) -- (0.5,0);}
            \onslide<3>{\draw[->, ultra thick] (0,0) -- (-0.5,1);}

		% Lattice generated
		\onslide<4->{
            \draw[->, ultra thick, myorange] (0,0) -- (0.5,0);
            \draw[->, ultra thick, myorange] (0,0) -- (-0.5,1);
            \foreach \x in {-1,-0.5,0,0.5,1,1.5,2}
                \foreach \y in {0,1,2}
                    \fill[myorange] (\x,\y) circle (1.1pt);
		\foreach \x in {-1,-0.5,0,0.5,1}
			\draw[-, myorange] (\x - 0.1, 2 + 0.2) -- (\x + 1 + 0.1, 0 - 0.2);
		\draw[-, myorange] (-1 - 0.1, 1 + 0.2) -- (-1 + 0.5 + 0.1, 0 - 0.2);
		\draw[-, myorange] (1.5 - 0.1, 2 + 0.2) -- (1.5 + 0.5 + 0.1, 1 - 0.2);
		\draw[-, myorange] (-1.15, 0) -- (2.15, 0);
		\draw[-, myorange] (-1.15, 1) -- (2.15, 1);
		\draw[-, myorange] (-1.15, 2) -- (2.15, 2);
		}

%		\node[anchor=south ] at (-0.5, 0) {\scriptsize \alert{$r^{-1}$}};
%		\node[anchor=south ] at (-0.5, 0.5) { \scriptsize \alert{$r^{-1} s$}};
%		\node[anchor=south ] at (-0.5, 1) {\scriptsize \alert{$r^{-1} s^2$}};
        \end{tikzpicture}
\end{figure}
\end{minipage}}
\end{frame}

%\begin{frame}
%\frametitle{Torus quotients}
%Suppose $T = (\mathbb{C}^\times)^d$ acts linearly on $\mathbb{C}^n$.
%\textbf{Goal}: understand $\mathbb{C}^n / T$. 
%
%\vspace{0.4cm}
%
%\begin{minipage}{0.5\textwidth}
%%\linespread{1}\selectfont  % Adjust the line spacing inside the minipage
%\onslide<2->{
%\textbf{Example}: $\mathbb{C}^\times \curvearrowright \mathbb{C}^2$:
%$$t \cdot (x, y) = (t x, t^{-1} y).$$
%}
%\onslide<3->{
%\hspace{-0.26cm}\textbf{Problem}: determine invariant polynomials
%\begin{align*}
%	&\,\,\,\,\,\,\,\,\, \mathbb{C}[x, y]^T \\ &= \{\text{polys } f : f(p) = f(T \cdot p)\}.
%\end{align*}
%}
%\end{minipage}
%\hfill
%\begin{minipage}{0.45\textwidth}
%\onslide<2->{\begin{tikzpicture}
%  % Draw axes
%  \draw[->] (-2.5, 0) -- (2.5, 0); % node[right] {$x$};
%  \draw[->] (0, -2) -- (0, 2); % node[above] {$y$};
%
%  % Plot the curve y^2 = x^3 - x + 1 manually
%  \draw[thick, myorange, samples=100, smooth, domain=0.5:2.4, variable=\x, <->] 
%    plot ({\x}, {1 / \x});
%  \draw[thick, myorange, samples=100, smooth, domain=-2.4:-0.5, variable=\x, <->] 
%    plot ({\x}, {1 / \x});
%  \draw[thick, myteal, samples=100, smooth, domain=0.1:2.4, variable=\x, <->] 
%    plot ({\x}, {1 / (5 *\x)});
%  \draw[thick, myteal, samples=100, smooth, domain=-2.4:-0.1, variable=\x, <->] 
%    plot ({\x}, {1 / (5 *\x)});
%  \draw[thick, myblue, samples=100, smooth, domain=0.5:2.4, variable=\x, <->] 
%    plot ({\x}, {- 1 / \x});
%  \draw[thick, myblue, samples=100, smooth, domain=-2.4:-0.5, variable=\x, <->] 
%    plot ({\x}, {- 1 / \x});
%  \draw[thick, blue, samples=100, smooth, domain=0.1:2.4, variable=\x, <->] 
%    plot ({\x}, {- 1 / (5 *\x)});
%  \draw[thick, blue, samples=100, smooth, domain=-2.4:-0.1, variable=\x, <->] 
%    plot ({\x}, {- 1 / (5 *\x)});
%
%   \filldraw[myorange] (2, 0.5) circle (2pt) node[above ] {\,\,\,$p$};
%   \node[myorange] at (1.3, 1.5) {$T \cdot p$};
%\end{tikzpicture}}
%\end{minipage}
%\end{frame}

\begin{frame}
\frametitle{Torus quotients}
\pause
Suppose $T = (\mathbb{C}^\times)^d$ acts linearly on $\mathbb{C}^n$.
\textbf{Goal}: understand $\mathbb{C}^n / T$. 

\pause
\textbf{Example}: $\mathbb{C}^\times \curvearrowright \mathbb{C}^2$ by $t \cdot (x, y) = (t x, t^{-1} y).$ 

\vspace{0.4cm}

\centerline{\begin{tikzpicture}[scale=1]
  % Draw axes
  \draw[->] (-2.5, 0) -- (2.5, 0); % node[right] {$x$};
  \draw[->] (0, -2) -- (0, 2); % node[above] {$y$};

  % Plot the curve y^2 = x^3 - x + 1 manually
  \draw[thick, myorange, samples=100, smooth, domain=0.5:2.4, variable=\x, <->] 
    plot ({\x}, {1 / \x});
  \draw[thick, myorange, samples=100, smooth, domain=-2.4:-0.5, variable=\x, <->] 
    plot ({\x}, {1 / \x});
  \draw[thick, myteal, samples=100, smooth, domain=0.1:2.4, variable=\x, <->] 
    plot ({\x}, {1 / (5 *\x)});
  \draw[thick, myteal, samples=100, smooth, domain=-2.4:-0.1, variable=\x, <->] 
    plot ({\x}, {1 / (5 *\x)});
  \draw[thick, myblue, samples=100, smooth, domain=0.5:2.4, variable=\x, <->] 
    plot ({\x}, {- 1 / \x});
  \draw[thick, myblue, samples=100, smooth, domain=-2.4:-0.5, variable=\x, <->] 
    plot ({\x}, {- 1 / \x});
  \draw[thick, blue, samples=100, smooth, domain=0.1:2.4, variable=\x, <->] 
    plot ({\x}, {- 1 / (5 *\x)});
  \draw[thick, blue, samples=100, smooth, domain=-2.4:-0.1, variable=\x, <->] 
    plot ({\x}, {- 1 / (5 *\x)});

   \filldraw[myorange] (2, 0.5) circle (2pt) node[above ] {\,\,\,$p$};
   \node[myorange] at (1.3, 1.5) {$T \cdot p$};
\end{tikzpicture}}

\pause
\hspace{-0.26cm}\textbf{Problem}: determine invariant polynomials
\begin{align*}
	&\mathbb{C}[x, y]^T = \{\text{polys } f : f(p) = f(T \cdot p)\}.
\end{align*}
\end{frame}

\begin{frame}
\frametitle{Torus quotients as toric varieties}
\onslide<2->{\begin{theorem}
$\mathbb{C}^n / T$ is a toric variety whose cone can be explicitly computed.
\end{theorem}
\textbf{Idea}:}
\onslide<3->{Invariant monomials are parametrised by points in $\mathbb{Z}^n$ satisfying linear equations.}
\onslide<4->{For $\mathbb{C}^\times \curvearrowright \mathbb{C}^2$:
$$x^k y^\ell = t^{k-\ell} x^k y^\ell \qquad \iff \qquad \textcolor{myorange}{k =\ell}.$$}
\onslide<5->{Choose $\sigma$ so $\sigma^\vee$ only sees points with non-negative coordinates.} %$(k, l) \in \mathbb{Z}^2_{\ge 0}.$}
\begin{figure}[H]
    \centering
    \begin{tikzpicture}[scale=1]
        % Left diagram
        \onslide<5->{\begin{scope}[shift={(-2,0)}]
            % Shaded area
            \fill[gray!30] (0,0) -- (0, 1.6) -- (1.6,1.6) -- (1.6,0) -- cycle;
            
            % Axes
            \draw[->] (-1.75,0) -- (1.75,0);
            \draw[->] (0,-1.75) -- (0,1.75);
            
            % Lattice points
            \foreach \x in {-1.5,-1,-0.5,0,0.5,1,1.5}
                \foreach \y in {-1.5,-1,-0.5,0,0.5,1,1.5}
                    \fill (\x,\y) circle (1pt);

            % Vectors
	  \draw[<->, thick, blue] (-1.75,-1.75) -- (1.75,1.75);
            %\draw[->, ultra thick] (0,0) -- (-0.5,1);
            \node at (.25, 1.25) {\small{$\sigma$}};
        \end{scope}}

        % Right diagram
        \onslide<4->{\begin{scope}[shift={(2,0)}]
            % Shaded area
            \onslide<5->{\fill[gray!30] (0,0) -- (0, 1.6) -- (1.6,1.6) -- (1.6,0) -- cycle;}
            
            % Axes
            \draw[->] (-1.75,0) -- (1.75,0);
            \draw[->] (0,-1.75) -- (0,1.75);
            
            % Lattice points
            \foreach \x in {-1.5,-1,-0.5,0,0.5,1,1.5}
                \foreach \y in {-1.5,-1,-0.5,0,0.5,1,1.5}
                    \fill (\x,\y) circle (1pt);

            % Vectors
	  \draw[<->, thick, myorange] (-1.75,-1.75) -- (1.75,1.75);
	  \foreach \x in {-1.5,-1,-0.5,0,0.5,1,1.5}
		\fill[myorange] (\x, \x) circle (2pt);
            %\draw[->, ultra thick] (0,0) -- (0.5,0);
            %\draw[->, ultra thick] (0,0) -- (-0.5,1);


            \onslide<5->{\node at (.25, 1.25) {\, \small{$\sigma^\vee$}};
	   \node[anchor= north west] at (0.5, 0.5) { \scriptsize \alert{$xy$}};
	   \node[anchor=north west] at (1, 1) {\!\!\!\! \scriptsize \alert{$x^2 y^2$}};}
	   %\node[anchor=north west] at (1.5, 1.5) {\!\!\!\! \scriptsize \alert{$x^3 y^3$}};}
        \end{scope}}
    \end{tikzpicture}
\end{figure}
\end{frame}


\begin{frame}
\frametitle{References}
%\begin{adjustwidth}{-1.7em}{-1.7em}
\begin{enumerate}%[leftmargin=-1cm]
\item[]
Stephen Boyd and Lieven Vandenberghe,
\emph{Convex optimization},
Cambridge University Press, 2004.

\vspace{0.5cm}

\item[]
William Fulton,
\emph{Introduction to toric varieties},
Princeton University Press, 1993.

\vspace{0.5cm}

\item[]
Shigeru Mukai,
\emph{An introduction to invariants and moduli},
Cambridge University Press, 2003.

\vspace{0.5cm}

\item[]
Miles Reid,
\emph{Undergraduate algebraic geometry},
Cambridge Univeristy Press, 1988.
\end{enumerate}
%\end{adjustwidth}
\end{frame}
\end{document}