\documentclass[12pt]{amsart}

\usepackage{amsmath}
\usepackage{amssymb}
\usepackage{amsfonts}
\usepackage[alphabetic]{amsrefs}
\usepackage{amsthm}
\usepackage{enumitem}
\usepackage{fullpage}
\usepackage{graphicx}
\usepackage{caption}
\usepackage{color}
\usepackage{hyperref}
\usepackage{mathtools}
\usepackage{stmaryrd} % for GIT //

\newcommand{\An}{\mathbb{A}^n}
\newcommand{\Pn}{\mathbb{P}^n}
\renewcommand{\a}{\mathfrak{a}}
\newcommand{\C}{\mathbb{C}}
\newcommand{\g}{\mathfrak{g}}
\newcommand{\h}{\mathfrak{h}}

\newtheorem{theorem}{Theorem}%[subsection]
\newtheorem{definition}[theorem]{Definition}
\newtheorem{corollary}[theorem]{Corollary}
\newtheorem{lemma}[theorem]{Lemma}
\newtheorem*{exercise}{Exercise}
\newtheorem{proposition}[theorem]{Proposition}
\newtheorem{claim}[theorem]{Claim}

\theoremstyle{remark}
\newtheorem*{remark}{Remark}

\theoremstyle{remark}
\newtheorem*{example}{Example}

\theoremstyle{remark}
\newtheorem*{moreexamples}{More examples}

\allowdisplaybreaks

\title{Checking Stability using the Hilbert-Mumford Criterion}
\author{Declan Fletcher}
\date{}

\begin{document}

\maketitle

\section*{The $G=SL_3$ case}

Let $G = SL_3(\C)$ and consider the adjoint action of $G$ on $\mathfrak{g} = \mathfrak{sl}_3(\C)$.
A one-parameter subgroup $\lambda$ of $T$ is of the form
$$t \mapsto \begin{pmatrix} t^a & & \\ & t^b & \\ & & t^c \end{pmatrix}$$
for some $a, b, \in \mathbb{Z}$ and $a + b + c = 0$.
Composing with the adjoint action, we have
$$t 
\overset{\lambda}{\mapsto} \begin{pmatrix} t^a & & \\ & t^b & \\ & & t^c \end{pmatrix}
 \overset{\mathrm{Ad}}{\mapsto} \begin{pmatrix} v_{11} & t^{a-b} v_{12} & t^{2a+b} v_{13} \\ t^{-a+b} v_{21} & v_{22} & t^{a+2b} v_{23} \\ t^{-2a-b} v_{31} & t^{-a-2b} v_{32} & v_{33} \end{pmatrix}.$$
So the representation of $\C^\times$ given by the composition $\mathrm{Ad} \circ \lambda: \C^\times \to GL(\mathfrak{g})$ decomposes into the following weight spaces:
\begin{align*}
	\g &= \h \oplus \g_{\varepsilon_1} \oplus \g_{\varepsilon_2} \oplus \g_{\varepsilon_1 + \varepsilon_2} \oplus \g_{-\varepsilon_1} \oplus \g_{-\varepsilon_2} \oplus \g_{-\varepsilon_1 - \varepsilon_2}. 
\end{align*}
The respective weights are
$$0, \, a-b, \, a+2b, \, 2a+b, \, -a+b, \, -a-2b, \,-2a-b.$$
Here $\varepsilon_1(t) = t^{a-b}, \varepsilon_2(t) = t^{a+2b}$.

Let us say the weights of $v \in \g$ are the weights in which $v$ has a non-zero component.
The Hilbert-Mumford criterion gives us conditions for the stability of $v$ in terms of the weights of $v$:
\begin{itemize}[label=-]
\item
$v \in \g$ is unstable if and only if there exists $\lambda \in X_*(T)\setminus\{0\}$ such that $v$ admits only positive or only negative weights.

\item
$v \in \g$ is semi-stable if and only if for all $\lambda \in X_*(T)\setminus\{0\}$, $v$ admits both non-negative and non-positive weights.

\item
$v \in \g$ is stable if and only if for all $\lambda \in X_*(T)\setminus\{0\}$, $v$ admits both positive and negative weights.
\end{itemize}

\begin{example}
The vector 
$$v = \begin{pmatrix} 0 & 0 & 1 \\ 0 & 0 & 1 \\ 0 & 0 & 0 \end{pmatrix} \in \g_{\varepsilon_2} \oplus \g_{\varepsilon_1 + \varepsilon_2}$$
has weights $2a + b, a + 2b$.
When $a = b = -1$, $v$ admits only negative weights, so $v$ is unstable.

The vector 
$$v = \begin{pmatrix} 0 & 1 & 0 \\ 1 & 0 & 0 \\ 0 & 0 & 0 \end{pmatrix} \in \g_{\varepsilon_1} \oplus \g_{-\varepsilon_1}$$
has weights $a-b, - (a-b)$.
For all $a, b \in \mathbb{Z}$, one of the weights is non-negative and one is non-positive, so $v$ is semistable.
However, $v$ is not stable since $a=b=1$ gives a one-parameter subgroup where we do not have both positive and negative weights.

The vector 
$$v = \begin{pmatrix} 0 & 1 & 0 \\ 1 & 0 & 1 \\ 0 & 1 & 0 \end{pmatrix} \in  \g_{\varepsilon_1} \oplus \g_{-\varepsilon_1} \oplus  \g_{\varepsilon_2} \oplus \g_{-\varepsilon_2}$$
has weights $a-b, -(a-b), a+2b, -(a+2b)$.
For all $a, b \in \mathbb{Z}$, either $a-b, -(a-b)$ is a pair of positive and negative weights, or $a+2b, -(a+2b)$ is, so $v$ always has positive and negative weights.
Therefore, $v$ is stable.
\end{example}

\begin{moreexamples}
We list further examples of the stability of vectors in $\mathfrak{sl}_3$, which were checked using a computer.
The following vectors are unstable:
$$
\begin{pmatrix} 
	0 & 1 & 0 \\ 
	0 & 0 & 1 \\
	0 & 0 & 0 
\end{pmatrix}, \, 
\begin{pmatrix} 
	0 & 1 & 0 \\ 
	0 & 0 & 0 \\
	1 & 0 & 0 
\end{pmatrix}, \, 
\begin{pmatrix} 
	0 & 1 & 1 \\ 
	0 & 0 & 0 \\
	0 & 1 & 0 
\end{pmatrix}.
$$
It is interesting to note that these are all nilpotent matrices.
The following vectors are semi-stable and not stable:
$$
\begin{pmatrix} 
	0 & 1 & 0 \\ 
	1 & 0 & 1 \\
	0 & 0 & 0 
\end{pmatrix}, \, 
\begin{pmatrix} 
	0 & 0 & 0 \\ 
	0 & 0 & 1 \\
	0 & 1 & 0 
\end{pmatrix}, \, 
\begin{pmatrix} 
	0 & 1 & 1 \\ 
	0 & 0 & 1 \\
	0 & 1 & 0 
\end{pmatrix}.
$$
The following vectors are stable:
$$
\begin{pmatrix} 
	0 & 1 & 0 \\ 
	0 & 0 & 1 \\
	1 & 0 & 0 
\end{pmatrix}, \, 
\begin{pmatrix} 
	0 & 1 & 0 \\ 
	1 & 0 & 1 \\
	0 & 1 & 0 
\end{pmatrix}, \, 
\begin{pmatrix} 
	0 & 1 & 1 \\ 
	1 & 0 & 0 \\
	1 & 0 & 0 
\end{pmatrix}.
$$ \\
\end{moreexamples}

\section*{The $G = Sp_4$ case}
Now let $G = Sp_4(\C)$ and $\mathfrak{g} = \mathfrak{sp}_4(\C)$.
Composing a one-parameter subgroup of $T$ with the adjoint action gives a map
$$t 
\overset{\lambda}{\mapsto} 
\begin{pmatrix} 
	t^a & & & \\ 
	& t^b & & \\ 
	& & t^{-b} & \\
	& & & t^{-a} 
\end{pmatrix}
\overset{\mathrm{Ad}}{\mapsto} 
\begin{pmatrix} 
	v_{11} & t^{a-b} v_{12} & t^{a+b} v_{13} & t^{2a} v_{14} \\ 
	t^{-(a-b)} v_{21} & v_{22} & t^{2b} v_{23} & t^{a+b} v_{24} \\ 
	t^{-(a+b)} v_{31} & t^{-2b} v_{32} & v_{33} & t^{a-b} v_{34} \\
	t^{-2a} v_{41} & t^{-(a+b)} v_{42} & t^{-(a-b)} v_{43} & v_{44}
\end{pmatrix},$$
where $a, b \in \mathbb{Z}$.
So the representation $\mathrm{Ad} \circ \lambda : \C^\times \to \mathfrak{sp}_4$ decomposes as
$$
\mathfrak{g} 
= \h \oplus \g_{\varepsilon_1-\varepsilon_2} \oplus \g_{2 \varepsilon_2} \oplus \g_{\varepsilon_1+\varepsilon_2} \oplus \g_{2\varepsilon_1} 
\oplus \g_{-(\varepsilon_1-\varepsilon_2)} \oplus \g_{-2 \varepsilon_2} \oplus \g_{-(\varepsilon_1+\varepsilon_2)} \oplus \g_{-2\varepsilon_1}, 
$$
where the respective weights are
$$0, \,\, a-b, \,\, 2b, \,\, a+b, \,\, 2a, \,\, -(a-b), \,\, -2b, \,\, -(a+b), \,\, -2a.$$
Here $\varepsilon_1(t) = t^{a}$, $\varepsilon_2(t) = t^{b}$.

\begin{example}
The vector 
$$v = 
\begin{pmatrix}
	0 & 1 & 0 & 0 \\
	0 & 0 & 1 & 0 \\
	0 & 0 & 0 & 1 \\
	0 & 0 & 0 & 0 
\end{pmatrix} \in \g_{\varepsilon_1 - \varepsilon_2} \oplus \g_{2 \varepsilon_2}
$$
has weights $a-b, 2b$.
When $a = -2, b = -1$, $v$ admits only negative weights, so $v$ is unstable.

The vector 
$$v = 
\begin{pmatrix}
	0 & 1 & 0 & 0 \\
	1 & 0 & 0 & 0 \\
	0 & 0 & 0 & 1 \\
	0 & 0 & 1 & 0 
\end{pmatrix} \in \g_{\varepsilon_1 - \varepsilon_2} \oplus \g_{-(\varepsilon_1 -  \varepsilon_2)}
$$
has weights $a-b, -(a-b)$, which always has a non-negative and non-positive weight for all $a, b \in \mathbb{Z}$, so $v$ is semi-stable.
When $a=b=1$, we do not have a positive and negative weight, so $v$ is not stable.

The vector
$$v = 
\begin{pmatrix}
	0 & 1 & 0 & 0 \\
	1 & 0 & 1 & 0 \\
	0 & 1 & 0 & 1 \\
	0 & 0 & 1 & 0 
\end{pmatrix} \in \g_{\varepsilon_1 - \varepsilon_2} \oplus \g_{2 \varepsilon_2} \oplus \g_{-(\varepsilon_1 -  \varepsilon_2)} \oplus \g_{- 2\varepsilon_2} 
$$
has weights $a-b, 2b, -(a-b), -2b$, which contains positive and negative weights for all $a, b \in \mathbb{Z}$.
Therefore, $v$ is stable.
\end{example}

\begin{moreexamples}
The following examples of stability for vectors in $\mathfrak{sp}_4$ were checked using a computer.
The following vectors are unstable:
$$
\begin{pmatrix} 
	0 & 1 & 0 & 0 \\ 
	0 & 0 & 0 & 0 \\
	0 & 1 & 0 & 1 \\
	0 & 0 & 0 & 0
\end{pmatrix}, \, 
\begin{pmatrix} 
	0 & 0 & 0 & 0 \\ 
	0 & 0 & 1 & 0 \\
	1 & 0 & 0 & 0 \\
	0 & -1 & 0 & 0
\end{pmatrix}, \, 
\begin{pmatrix} 
	0 & 0 & 0 & 0 \\ 
	1 & 0 & 1 & 0 \\
	1 & 0 & 0 & 0 \\
	0 & -1 & 1 & 0
\end{pmatrix}.
$$
It is interesting to note that these are all nilpotent matrices.
The following vectors are semi-stable and not stable:
$$
\begin{pmatrix} 
	0 & 0 & 0 & 0 \\ 
	0 & 0 & 1 & 0 \\
	0 & 1 & 0 & 0 \\
	0 & 0 & 0 & 0
\end{pmatrix}, \, 
\begin{pmatrix} 
	0 & 0 & 0 & 0 \\ 
	1 & 0 & 1 & 0 \\
	0 & 1 & 0 & 0 \\
	0 & 0 & 1 & 0
\end{pmatrix}, \, 
\begin{pmatrix} 
	0 & 1 & 1 & 1 \\ 
	0 & 0 & 0 & -1 \\
	1 & 0 & 0 & 1 \\
	0 & -1 & 0 & 0
\end{pmatrix}.
$$
The following vectors are stable:
$$
\begin{pmatrix} 
	0 & 0 & 1 & 0 \\ 
	1 & 0 & 0 & -1 \\
	0 & 1 & 0 & 0 \\
	0 & 0 & 1 & 0
\end{pmatrix}, \,
\textcolor{red}{\begin{pmatrix} 
	0 & 0 & 1 & 0 \\ 
	0 & 0 & 0 & -1 \\
	0 & 1 & 0 & 0 \\
	1 & 0 & 0 & 0
\end{pmatrix}, \, 
\begin{pmatrix} 
	0 & 0 & 0 & 1 \\ 
	1 & 0 & 0 & 0 \\
	0 & 1 & 0 & 0 \\
	0 & 0 & 1 & 0
\end{pmatrix}.}
$$ \\
\end{moreexamples}
The two red matrices are counterexamples to our upcoming claim about the general case.
We have
$$
\begin{pmatrix} 
	0 & 0 & 1 & 0 \\ 
	0 & 0 & 0 & -1 \\
	0 & 1 & 0 & 0 \\
	1 & 0 & 0 & 0
\end{pmatrix} \in \g_{\varepsilon_1+\varepsilon_2} \oplus \g_{-2\varepsilon_2} \oplus \g_{-2\varepsilon_1}$$
with weights $a+b, -2b, -a$ and 
$$
\begin{pmatrix} 
	0 & 0 & 0 & 1 \\ 
	1 & 0 & 0 & 0 \\
	0 & 1 & 0 & 0 \\
	0 & 0 & 1 & 0
\end{pmatrix} \in \g_{2\varepsilon_1} \oplus \g_{-(\varepsilon_1 - \varepsilon_2)} \oplus \g_{-2\varepsilon_2}$$
with weights $2a, -(a-b), -2b$.

\section*{The general case}
Let $G$ be a connected complex reductive group, $\g$ the Lie algebra and $T$ a maximal torus.
We are interested in the adjoint action of $T$ on $\g$.

Choose $x_\alpha \in \g_\alpha$ for each $\alpha \in \Phi$ such that $x_\alpha$ is a basis for $\g_\alpha$.
For any $x \in \g$, we can write
$$x = x_0 + \sum_{\alpha \in \Phi} k_\alpha x_\alpha, \qquad k_\alpha \in \C.$$
Denote
$$S_x = \{\alpha \in \Phi \, | \, k_\alpha \ne 0\} \subseteq \Phi.$$

The list of examples leads to the following claim:

\begin{claim}
Let $x \in \g$.
Then:
\begin{enumerate}[label=(\roman*)]
\item
$x$ is unstable if and only if for all $Z \subseteq S_x$, 
$$\sum_{\alpha \in Z} \alpha \ne 0;$$

\item
$x$ is semi-stable if and only if there exists $Z \subseteq S_x$ such that
$$\sum_{\alpha \in Z} \alpha = 0;$$

\item
$x$ is stable if and only if there exists $Z \subseteq S_x$ with $|Z| > \mathrm{rank}(G) = \mathrm{rank}(Y(T))$ such that
$$\sum_{\alpha \in Z} \alpha = 0.$$
\end{enumerate}
\end{claim}

Clearly, (i) and (ii) are the same statement.
Suppose $x$ is unstable and $S_x = \{\alpha_1, \ldots, \alpha_m\}$.
The Hilbert-Mumford criterion tells us there exists $\lambda \in Y(T)$ such that 
$$\langle \lambda, \alpha_1 \rangle, \ldots, \langle \lambda, \alpha_r \rangle > 0.$$
Then for all $Z \subseteq S_x$,
$$\left\langle\lambda, \sum_{\alpha \in Z} \alpha \right \rangle
= \sum_{\alpha \in Z} \langle \lambda, \alpha \rangle > 0 \implies \sum_{\alpha \in Z} \alpha \ne 0$$
The converse statement needed to prove (i) and (ii) would be: 
suppose there exists $Z \subseteq S_x$ such that $\sum_{\alpha \in Z} \alpha = 0$; to see $x$ is semi-stable, by Hilbert-Mumford we need to show that for all $\lambda \in Y(T)$, there exists $\alpha_i, \alpha_j \in S_x$ such that $\langle \lambda, \alpha_i \rangle \le 0, \langle \lambda, \alpha_j \rangle \ge 0$.

Statement (iii) is not true due to the red counter examples above.
However, it may be salvageable if we allow multiples of roots to sum to zero.
For example, although the vector in $\g_{\varepsilon_1+\varepsilon_2} \oplus \g_{-2 \varepsilon_2} \oplus \g_{-2\varepsilon_1}$ doesn't have a subset of $S_x$ that sums to zero, we do have
$$2 (\varepsilon_1+\varepsilon_2) -2 \varepsilon_2 -2\varepsilon_1 = 0.$$

\end{document}