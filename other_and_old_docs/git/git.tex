\documentclass[12pt]{amsart}

\usepackage{amsmath}
\usepackage{amssymb}
\usepackage{amsfonts}
\usepackage[alphabetic]{amsrefs}
\usepackage{amsthm}
\usepackage{enumitem}
\usepackage{fullpage}
\usepackage{graphicx}
\usepackage{caption}
\usepackage{color}
\usepackage{hyperref}
\usepackage{mathtools}
\usepackage{stmaryrd} % for GIT //

\newcommand{\An}{\mathbb{A}^n}
\newcommand{\Pn}{\mathbb{P}^n}
\renewcommand{\a}{\mathfrak{a}}
\newcommand{\C}{\mathbb{C}}

\newtheorem{theorem}{Theorem}%[subsection]
\newtheorem{definition}[theorem]{Definition}
\newtheorem{corollary}[theorem]{Corollary}
\newtheorem{lemma}[theorem]{Lemma}
\newtheorem*{exercise}{Exercise}
\newtheorem{proposition}[theorem]{Proposition}

\theoremstyle{remark}
\newtheorem*{remark}{Remark}

\theoremstyle{remark}
\newtheorem*{example}{Example}

\allowdisplaybreaks

\title{Geometric Invariant Theory}
\author{Declan Fletcher}
\date{}

\begin{document}

\maketitle

Masoud has told me about a problem he is interested in in the field of \emph{geometric invariant theory}.
Here are some notes from our meeting: \\

Let $G$ be an algebraic group acting on a vector space $V$.
The motivating example is $G = SL_n$ acting on its Lie algebra $V = \mathfrak{sl}_n$ by conjugation.
We have the ideas of \emph{unstable vectors, semi-stable vectors} and \emph{stable vectors}:
\begin{itemize}
\item[-]
$v \in V$ is unstable if $0 \in \overline{G \cdot v}$.

\item[-]
$v \in V$ is semi-stable if it is not unstable.

\item[-]
$v \in V$ is stable if $G \cdot v$ is closed and $G_v = \mathrm{stab}_G(v)$ is finite.

\end{itemize}

\begin{example}
Let $G = \mathbb{C}^\times$ act on $V = \mathbb{C} \oplus \mathbb{C}$ by $g \cdot (x, y) = (gx, g^{-1} y)$.
The orbits are:
\begin{itemize}
\item[-]
$(0, 0)$, which is closed and unstable.

\item[-]
The $x$-axis without $(0, 0)$, which is not closed and unstable.

\item[-]
The $y$-axis without $(0, 0)$, which is not closed and unstable.

\item[-]
$\mathcal{O}_u := \{(x, y) \, | \, xy = u\}$ for each $u \in \mathbb{C}^\times$, which are closed, semi-stable and stable.
\end{itemize}
\end{example}

There's a second definition of unstable;
$v \in V$ is unstable if for all $f \in \C[V]^G$, $f(v) = f(0)$.
In the example above, we can consider $\mathbb{C}[V] = \mathbb{C}[x, y]$.

\begin{exercise}
Find $\mathbb{C}[V]^G$ for the above example, and check we find the same vectors are unstable.
\end{exercise}

There's also a third way to check if vectors are unstable, called the \emph{Hilbert-Mumford criterion}.

\begin{exercise}
Use the Hilbert-Mumford criterion to find the stable and semi-stable vectors for the above example.
\end{exercise}

In general, the case of $G$ acting on $V = \mathfrak{g}$ is quite well understood (exercise: find a reference).
Masoud is interested in understanding what happens when $T$, the subgroup of diagonal elements of $G$ (i.e., a torus), acts on $V$.

\begin{exercise}
Understand the cases for $G = SL_2, SL_3$ acting on $V = \mathfrak{sl_2}, \mathfrak{sl_3}$, and \emph{also the diagonal elements}.
The $n=2$ should reduce to the example above.
\end{exercise}

Generally, we are interested in the case of $T$ acting on $\mathfrak{sl}_n$, specifically:
\begin{itemize}[label=-]
\item
What is $\mathbb{C}[V]^T$?

\item
What are the unstable/semi-stable/stable points?

\item
Using the Hilbert-Mumford criterion to solve these.
\end{itemize}

\newpage
\subsection*{Using the second definition of stability}
We recall the second definition of unstable vectors; that $v \in V$ is unstable if for all $f \in \C[V]^G$, $f(v) = f(0)$.
We claim that for the above example of $G = \C^\times$ acting on $\C \oplus \C$ by $g \cdot (x, y) = (gx, g^{-1}y)$, the invariant ring is $\C[V]^G = \C[xy]$ (see the appendix for a proof of this fact).
We now use this to again check which orbits are unstable:
\begin{itemize}[label=-]
\item
The orbit $\{(0, 0)\}$ trivially satisfies the second definition of unstable.

\item
The orbit $G \cdot (x, 0)$ with $x \ne 0$ is unstable since for $p \in \C[V]^G = \C[xy]$, $p(x, 0) = p(0, 0)$.

\item
The orbit $G \cdot (0, y)$ with $y \ne 0$ is unstable is unstable for the same reason.

\item 
The orbit $\mathcal{O}_u = \{(x, y) \, | \, xy = u\}$ is semi-stable since $p(x, y) = xy \in \C[V]^G$ and $p(x, y) = u \ne p(0,0)$.
\end{itemize}

\subsection*{The Hilbert-Mumford criterion}
We state the Hilbert-Mumford criterion\footnote{see \url{https://en.wikipedia.org/wiki/Hilbert-Mumford\_criterion}} and use this to check the unstable vectors.
Let $G$ be a general reductive group acting linearly on a vector space $V$.
A \emph{1-parameter subgroup} of $G$ is a non-trivial morphism (to do: what does this mean exactly) $\lambda: \mathbb{G}_m = \C^\times \to G$.
This amounts to a finite dimensional representation of $\C^\times$, and we can decompose $V$ as $\bigoplus_i V_i$ where on each component the action is given as $\lambda(t) \cdot v = t^i v$.
The integer $i$ is called the weight.
The Hilbert-Mumford criterion then says:
\begin{itemize}[label=-]
\item
$v$ is unstable if and only if there is a 1-parameter subgroup of $G$ for which $v$ only admits positive or negative weights,

\item 
$v$ is semi-stable if and only if every 1-parameter subgroup there are both non-positive and non-negative weights,

\item
$v$ is stable if and only if for every 1-parameter subgroup there are both positive and negative weights.
\end{itemize}
An alternative statement\footnote{see section 3 of \url{https://www.staff.uni-mainz.de/pommeren/MathMisc/HilbMumf.pdf}} of the Hilbert-Mumford criterion is:
\begin{theorem}
Let $G$ be a connected reductive algebraic group, $T \le G$ a maximal torus, $V$ a rational $G$-module, and $v \in V$.
Then the following are equivalent:
\begin{enumerate}[label=(\roman*)]
\item
$x$ is unstable, i.e., $0 \in \overline{G \cdot x}$.

\item
The orbit $G \cdot x$ contains an element $y$ that is unstable for $T$, i.e., $0 \in \overline{T \cdot y}$.

\item
There is a 1-parameter subgroup $\lambda: \mathbb{G}_m \to G$ with $\lim_{t \to 0} \lambda(t) \cdot v = 0$.
\end{enumerate}
\end{theorem}

This second statement is easier to understand, the exact meaning of 1-parameter subgroup aside.

\begin{example}
In the case that $G = \C^\times$, I think a 1-parameter subgroup is just $\lambda: \C^\times \to \C^\times$, $\lambda(t) = t^{k}$ for some $k \in \mathbb{Z}$.
\begin{itemize}[label=-]
\item
The orbit $\{(0, 0)\}$ has $\lim_{t \to 0} \lambda(t) \cdot (0, 0) = (0, 0)$ for any 1-parameter subgroup.

\item
Let $\lambda(t) = t$, then $\lim_{t \to 0} \lambda(t) \cdot (x, 0) = \lim_{t \to 0} (t x, 0) = (0, 0)$ shows $(x, 0)$ is unstable.

\item
Let $\lambda(t) = t^{-1}$, then $\lim_{t \to 0} \lambda(t) \cdot (0, y) = \lim_{t \to 0} (0, ty) = (0, 0)$ shows $(0, y)$ is unstable.

\item 
Let $\lambda(t) = t^k$ with $k \in \mathbb{Z}$ be any 1-parameter subgroup.
Then for $(x, y) \in \mathcal{O}_u$, i.e., $xy = u \ne 0$, 
$$\lim_{t \to 0} \lambda(t) \cdot (x, y) = \lim_{t \to 0} (t^k x, t^{-k} y) = (0, \infty) \ne (0, 0).$$
So $(x, y)$ is not unstable and is semi-stable.
\end{itemize}
\end{example}

\section*{$SL_2$ and its diagonal elements acting on $\mathfrak{sl}_2$}
Let $G = SL_2$, $V = \mathfrak{sl}_2$ and $T \le G$ be the subgroup of diagonal elements.
Then $T$ acts on $V$ by conjugation.
We write elements of $T$ and $V$ as
$$
g = \begin{pmatrix} \lambda & 0 \\ 0 & \lambda^{-1} \end{pmatrix} \in T,
\qquad
v = \begin{pmatrix} x & y \\ z & -x \end{pmatrix},$$
and we get
$$g \cdot v = \begin{pmatrix} x & \lambda^2 y \\ \lambda^{-2} z & -x \end{pmatrix}.$$

\subsection*{Checking stability with the first definition}
There are a set of orbits for each $x \in \C$, namely
\begin{itemize}[label=-]
\item $\{(x, 0, 0)\}$, which is closed, and unstable if $x = 0$, and semi-stable but not stable otherwise.
\item $\{(x, y, 0) : y \in \C^\times\}$, which is not closed and unstable if $x=0$, and semi-stable but not stable otherwise.
\item $\{(x, 0, z) : z \in\C^\times\}$, which is not closed and unstable if $x=0$, and semi-stable but not stable otherwise.
\item $\mathcal{O}_u = \{(x, y, z) : yz = u\}$ for each $u \in \C^\times$.
These orbits are all closed, and also semi-stable and stable.
\end{itemize}


\subsection*{Checking stability with the second definition}
In the appendix, we compute that $\C[V]^T = \C[x, yx]$.
Let's apply the second definition of stability to check which orbits are stable ($v$ is unstable if $f(v) = f(0)$ for all $f \in \C[V]^T$);
\begin{itemize}[label=--]
\item with $x = 0$, we have the orbits;
	\begin{itemize}
	\item $\{0\}$ is clearly unstable using the second definition.
	\item $\{(0, y, 0) : y \in \C^\times\}$ is unstable since for all $p \in \C[x, yz]$, $p(0, y, 0) = p(0, 0, 0)$.
	\item $\{(0, 0, z) : z \in \C^\times\}$ is unstable since for all $p \in \C[x, yz]$, $p(0, 0, z) = p(0, 0, 0)$.
	\item For each $u\in \C^\times$, $\mathcal{O}_u = \{(0, y, z) : yz = u\}$. These orbits are semi-stable since $p(x,y,z) = yz \in \C[V]^T$ and $p(0, y, z) = u \ne 0 = p(0, 0, 0)$.
	\end{itemize}
\item for each $x \ne 0$;
	\begin{itemize}
	\item $\{(x, 0, 0)\}$ is semi-stable since $p(x, y, z) = x \in \C[V]^T$ and $p(x, 0, 0) = x \ne 0 = p(0, 0, 0)$.
	\item $\{(x, y, 0) : y \in \C^\times\}$ is semi-stable since $p(x, y, z) = x \in \C[V]^T$ and $p(x, y, 0) = x \ne 0 = p(0, 0, 0)$.
	\item $\{(x, 0, z) : z \in \C^\times\}$ is semi-stable since $p(x, y, z) = x \in \C[V]^T$ and $p(x, y, 0) = x \ne 0 = p(0, 0, 0)$.
	\item For each $u\in \C^\times$, $\mathcal{O}_u = \{(x, y, z) : yz = u\}$.
	This orbit is semi-stable since $p(x, y, z) = x \in \C[V]^T$ and $p(x, y, z) = x \ne 0 = p(0, 0, 0)$.
	\end{itemize}
\end{itemize}

\subsection*{Checking stability with the Hilbert-Mumford criterion}
In this case, a 1-parameter subgroup of $T$ is a map $\lambda_k : \C^\times \to T$ which must be of the form
$$\lambda_k(t) = \begin{pmatrix} t^k & 0 \\ 0 & t^{-k} \end{pmatrix}$$
for some $k \in \mathbb{Z}\setminus \{0\}$.
\begin{itemize}[label=--]
\item with $x = 0$, we have the orbits;
	\begin{itemize}
	\item $\{0\}$ is unstable as $\lim_{t\to 0} \lambda_1(t) \cdot 0 = \lim_{t \to 0} 0 = 0$.
	\item $\{(0, y, 0) : y \in \C^\times\}$ is unstable since $\lim_{t \to 0} \lambda_1(t) \cdot (0, y, 0) = \lim_{t \to 0} (0, t^2 y, 0) = 0$.
	\item $\{(0, 0, z) : z \in \C^\times\}$ is unstable since $\lim_{t \to 0} \lambda_{-1}(t) \cdot (0, 0, z) = \lim_{t \to 0} (0, 0, t^2 z) = 0$.
	\item For each $u\in \C^\times$, $\mathcal{O}_u = \{(0, y, z) : yz = u\}$. These orbits are semi-stable since for any 1-parameter subgroup $\lambda_k$, $\lim_{t \to 0} \lambda_k(t) \cdot (0, y, z) = \lim_{t \to 0} (0, t^{2k} y, t^{-2k} z) \ne 0$.
	\end{itemize}
\item for each $x \ne 0$;
	\begin{itemize}
	\item $\{(x, 0, 0)\}$ is semi-stable since for any 1-parameter subgroup $\lambda_k$, $\lim_{t \to 0} \lambda_k(t) \cdot (x, 0, 0) = \lim_{t \to 0} (x, 0, 0) \ne 0$.
	\item $\{(x, y, 0) : y \in \C^\times\}$ is semi-stable since for any 1-parameter subgroup $\lambda_k$, $\lim_{t \to 0} \lambda_k(t) \cdot (x, y, 0) = \lim_{t \to 0} (x, t^{2k} y, 0) \ne 0$.
	\item $\{(x, 0, z) : z \in \C^\times\}$ is semi-stable since for any 1-parameter subgroup $\lambda_k$, $\lim_{t \to 0} \lambda_k(t) \cdot (x, 0, z) = \lim_{t \to 0} (x, 0, t^{-2k} z) \ne 0$.
	\item For each $u\in \C^\times$, $\mathcal{O}_u = \{(x, y, z) : yz = u\}$.
	This orbit is semi-stable since for any 1-parameter subgroup $\lambda_k$, $\lim_{t \to 0} \lambda_k(t) \cdot (x, y, z) = \lim_{t \to 0} (x, t^{2k} y, t^{-2k} z) \ne 0$.
	\end{itemize}
\end{itemize}

\subsection*{The map $\mathfrak{g}\sslash T \to \mathfrak{g}\sslash G$} 
Masoud writes that we have a canonical map $\mathfrak{g}\sslash T \to \mathfrak{g}\sslash G$ \footnote{see \url{https://mathoverflow.net/questions/463567/git-quotient-of-a-reductive-lie-algebra-by-the-maximal-torus}}.
How does this correspond to a map between the invariant rings $\C[V]^T = \C[x, yz]$ and $\C[V]^G=\C[x^2 + yz]$, and how can we see this explicitly in this case?


\section*{Type $A_2$}
Let $G = SL_3$, and $T \le G$ be the torus
$$T = \left\{\begin{pmatrix} \lambda & & \\ & \mu & \\ & & \lambda^{-1} \mu^{-1} \end{pmatrix} : \lambda, \mu \in\C^\times \right\}.$$
Then $T$ acts on $V = \mathfrak{sl}_3$ by conjugation, i.e., if
$$g = \begin{pmatrix} \lambda & & \\ & \mu & \\ & & \lambda^{-1} \mu^{-1} \end{pmatrix}, \quad v = \begin{pmatrix} a_{11} & a_{12} & a_{13} \\ a_{21} & a_{22} & a_{23} \\ a_{31} & a_{32} & a_{33} \end{pmatrix},$$
then
\begin{align*}
g \cdot v = g v g^{-1} &= \begin{pmatrix} \lambda & & \\ & \mu & \\ & & \lambda^{-1} \mu^{-1} \end{pmatrix} \begin{pmatrix} a_{11} & a_{12} & a_{13} \\ a_{21} & a_{22} & a_{23} \\ a_{31} & a_{32} & a_{33} \end{pmatrix} \begin{pmatrix} \lambda^{-1} & & \\ & \mu^{-1} & \\ & & \lambda \mu \end{pmatrix} \\
&= 
\begin{pmatrix}
a_{11} & \lambda \mu^{-1} a_{12} & \lambda^2 \mu a_{13} \\
\lambda^{-1} \mu a_{21} & a_{22} & \lambda \mu^2 a_{23} \\
\lambda^{-2} \mu^{-1} a_{31} & \lambda^{-1} \mu^{-2} a_{32} & a_{33}
\end{pmatrix}
\end{align*}

Let
$$R = \C[a_{11}, a_{22}, a_{12} a_{21}, a_{23} a_{32}, a_{13} a_{31}, a_{12} a_{23} a_{31}, a_{13} a_{21} a_{32}].$$
I claim that $\C[V]^T = R$.
The $\supseteq$ inclusion is straightforward to check by computing the action of $g \in T$ on each generator and noting they are invariant.
To prove the other inclusion, first note that a general monomial in $R$ is of the form
\begin{align*}
	&\quad \,\, a_{11}^k a_{22}^l (a_{12} a_{21})^{i_1} (a_{23} a_{32})^{i_2} (a_{13} a_{31})^{i_3} (a_{12} a_{23} a_{31})^{i_4} (a_{13} a_{21} a_{32})^{i_5} \\
	&= a_{11}^k a_{22}^l a_{12}^{i_1 + i_4} a_{13}^{i_3 + i_5} a_{21}^{i_1+i_5} a_{23}^{i_2 + i_4} a_{31}^{i_3 + i_4} a_{32}^{i_2+i_5}.
\end{align*}
for some $k, l, i_1, \ldots, i_5 \in \mathbb{Z}_{\ge 0}$.
A general $q \in \C[V]$ is of the form
$$q = \sum q_{m,n,j_1,\ldots,j_6} a_{11}^m a_{22}^n a_{12}^{j_1} a_{13}^{j_2} a_{21}^{j_3} a_{23}^{j_4} a_{31}^{j_5} a_{32}^{j_6},$$
where the sum is over a finite number of tuples of exponents $(m, n, j_1, \ldots j_6)$.
Now assume that $q$ is invariant, so $q(v) = q(g \cdot v)$ for all $g \in T$ and $v \in V$;
\begin{align*}
	&\quad \, \, \sum q_{m,n,j_1,\ldots,j_6} a_{11}^m a_{22}^n a_{12}^{j_1} a_{13}^{j_2} a_{21}^{j_3} a_{23}^{j_4} a_{31}^{j_5} a_{32}^{j_6} \\
	&= q(g \cdot v) \\
	&= \sum q_{m,n,j_1,\ldots,j_6} a_{11}^m a_{22}^n (\lambda \mu^{-1} a_{12})^{j_1} (\lambda^2 \mu a_{13})^{j_2} (\lambda^{-1} \mu a_{21})^{j_3} (\lambda \mu^2 a_{23})^{j_4} (\lambda^{-2} \mu^{-1} a_{31})^{j_5} (\lambda^{-1} \mu^{-2} a_{32})^{j_6} \\
	&= \sum q_{m,n,j_1,\ldots,j_6} \lambda^{j_1 + 2j_2 - j_3 + j_4 -2 j_5 - j_6} \mu^{- j_1 +j_2 +j_3 + 2 j_4 - j_5 -2 j_6} a_{11}^m a_{22}^n a_{12}^{j_1} a_{13}^{j_2} a_{21}^{j_3} a_{23}^{j_4} a_{31}^{j_5} a_{32}^{j_6} 
\end{align*}
or equivalently, by equating coefficients that
$$ (\lambda^{j_1 + 2j_2 - j_3 + j_4 -2 J_5 - j_6} \mu^{- j_1 +j_2 +j_3 + 2 j_4 - j_5 -2 j_6}  - 1) q_{m, n, j_1, \ldots, j_6} = 0$$
for each choice of exponents $(m, n, j_1, \ldots, j_6)$ in the sum.

To show our desired inclusion, we need to show that if $q$ is invariant and has a monomial with exponents $(m, n, j_1, \ldots j_6)$ which are not of the form
\begin{align*}
	m &= k & j_3 &= i_1 + i_5 \\
	n &= l & j_4 &= i_2 + i_4 \\
	j_1 &= i_1 + i_4 & j_5 &= i_3 + i_4 \\
	j_2 &= i_3 + i_5 & j_6 &= i_2 + i_5
\end{align*}
for some $k, l, i_1, \ldots, i_5 \in \mathbb{Z}_{\ge 0}$ (i.e., a monomial not in $R$), then 
$$\lambda^{j_1 + 2j_2 - j_3 + j_4 -2 j_5 - j_6} \mu^{- j_1 +j_2 +j_3 + 2 j_4 - j_5 -2 j_6}  - 1 \ne 0$$
for some $\lambda, \mu \in \C^\times$, so that $q_{m, n, j_1, \ldots, j_6} = 0$ and in fact we have $q \in R$.
We'll show the equivalent statement that if $j_1, \ldots, j_6 \in \mathbb{Z}_{\ge 0}$ such that
\begin{equation}
\begin{aligned}
	j_1 + 2j_2 - j_3 + j_4 -2 j_5 - j_6 &= 0 \\
	- j_1 +j_2 +j_3 + 2 j_4 - j_5 -2 j_6 &= 0
\end{aligned}
\label{eq1}
\end{equation}
then there exist $i_1, \ldots, i_5 \in \mathbb{Z}_{\ge 0}$ such that 
\begin{equation}
\begin{aligned}
	j_1 &= i_1 + i_4  & \qquad & j_4 &= i_2 + i_4 \\
	j_2 &= i_3 + i_5 & \qquad & j_5 &= i_3 + i_4 \\
	j_3 &= i_1 + i_5 & \qquad & j_6 &= i_2 + i_5
\end{aligned}
\label{eq2}
\end{equation}
System \ref{eq1} implies the following equalities
\begin{align*}
	j_3 + j_4 - j_1 &= j_6 \\
	j_1 - j_3 = j_4 - j_6 &= j_5 - j_2 \\
	- j_1 - j_2 + j_3 + j_5 &= 0 \\
	j_1 + j_6 - j_3 - j_4 &= 0
\end{align*}
After reducing system \ref{eq2} to echelon form, one can check that a solution is
\begin{align*}
	i_1 &= j_3 - i_5 \\
	i_2 &= j_6 - i_5 \\
	i_3 &= j_2 - i_5 \\
	i_4 &= j_1 - j_3 + i_5
\end{align*}
where $i_5$ is arbitrary.
However, we want to chooce $i_5 \in \mathbb{Z}_{\ge 0}$ such that $i_1, \ldots, i_4 \ge 0$ also.
I claim that $i_5 = \mathrm{min}\{j_2, j_3, j_6\} \ge 0$ works.
Then,
\begin{align*}
	i_1 &= j_3 - i_5 \ge 0 \\
	i_2 &= j_6 - i_5 \ge 0 \\
	i_3 &= j_2 - i_5 \ge 0 \\
	i_4 
	&= j_1 - j_3 + i_5
	= \begin{cases}
				j_1 - j_3 + j_3 = j_1 \ge 0 & \text{if } i_5 = j_3 \\
				j_5 - j_2 + j_2 = j_5 \ge 0 & \text{if } i_5 = j_2 \\
				j_4 - j_6 + j_6 = j_1 \ge 0 & \text{if } i_5 = j_6 \\
		\end{cases}
\end{align*}
where we have used the equality $j_1 - j_3 = j_4 - j_6 = j_5 - j_2$ to check the sign of $i_4$.
This completes the proof that $\C[V]^T = R$.

A next natural question to ask is which vectors are unstable and stable, which we can now check using this invariant ring.
(Note that there are too many orbits to systematically check all of them.)

Let 
$$v = \begin{pmatrix} 0 & 1 & 0 \\ 0 & 0 & 0 \\ 0 & 0 & 0 \end{pmatrix}.$$
Then
$$T \cdot v
= \left\{
\begin{pmatrix} 0 & a_{12} & 0 \\ 0 & 0 & 0 \\ 0 & 0 & 0 \end{pmatrix} : a_{12} \in \C^\times
\right\}.$$
This this is unstable, since $0 \in \overline{T \cdot v}$.
This also tells us this orbit is not closed.
We can also see that this orbit is unstable since if $p(v) = p(0)$ for all $p \in \C[V]^T$.

Fix $u \in \C^\times$ and now let
$$v = \begin{pmatrix} 0 & u & 0 \\ 1 & 0 & 0 \\ 0 & 0 & 0 \end{pmatrix}.$$
Then
$$T \cdot v 
= \left\{
\begin{pmatrix} 0 & a_{12} & 0 \\ a_{21} & 0 & 0 \\ 0 & 0 & 0 \end{pmatrix} : a_{12} a_{21} = u
\right\}.$$
Let $p(w) = a_{12} a_{21} \in \C[V]^T$.
Then $p(v) = u \ne 0 = p(0)$, so $v$ is semi-stable.
However,
$$T_v = \left\{
\begin{pmatrix} \lambda & 0 & 0 \\ 0 & \lambda & 0 \\ 0 & 0 & 1 \end{pmatrix} : \lambda \in \C^\times
\right\},$$
which is not finite, so $v$ is not stable.
The orbit is closed.
This is interesting as it gives a vector which is semi-stable but not stable, of which there were none in the $A_1$ case.

Again fix $u \in \C^\times$ and let 
$$v = \begin{pmatrix} 0 & u & 0 \\ 0 & 0 & 1 \\ 1 & 0 & 0 \end{pmatrix}.$$
Then
$$T \cdot v = \left\{
\begin{pmatrix} 0 & a_{12} & 0 \\ 0 & 0 & a_{23} \\ a_{31} & 0 & 0 \end{pmatrix} : a_{12} a_{23} a_{31} = u
\right\}.$$
This orbit is in fact stable, as it is closed and $T_v = \{1\}$.

Note that
\begin{align*}
	\C[V]^T &= \C[a_{11}, a_{22}, a_{12} a_{21}, a_{23} a_{32}, a_{13} a_{31}, a_{12} a_{23} a_{31}, a_{13} a_{21} a_{32}] \\
	&= \C[x, y, z, t, u, v, w] / (z t u - v w)
\end{align*}


\section*{Type $C_2$}
Let $G = Sp_4 = \{
g \in GL_n : g^t M g = M
\}
$,
where
$$M = \begin{pmatrix} & & & 1 \\ & & -1 & \\ & 1 & & \\ -1 & & & \end{pmatrix}.$$
Then one can show that
$$T = \left\{
\begin{pmatrix} \lambda & & & \\ & \mu & & & \\ & & \mu^{-1} & \\ & & & \lambda^{-1} \end{pmatrix} :
\lambda, \mu \in \C^\times 
\right\}$$
is a maximal torus.
$T$ acts on $V = \mathfrak{sp}_4$, i.e.,
$$\mathfrak{sp}_4
= \left\{
\begin{pmatrix}
a_{11} & a_{12} & a_{13} & a_{14} \\
a_{21} & a_{22} & a_{23} & - a_{13} \\
a_{31} & a_{32} & - a_{22} & a_{12} \\
a_{41} & - a_{31} & a_{21} & - a_{11}
\end{pmatrix} : 
a_{ij} \in \mathbb{C} \text{ for all } i, j
\right\}.$$
For $g \in T$ and $v \in V$, we have
$$g \cdot v = g v g^{-1} 
= 
\begin{pmatrix}
a_{11} & \lambda \mu^{-1} a_{12} & \lambda \mu a_{13} & \lambda^2 a_{14} \\
\lambda^{-1} \mu a_{21} & a_{22} & \mu^2 a_{23} & - \lambda \mu a_{13} \\
\lambda^{-1} \mu^{-1} a_{31} & \mu^{-2} a_{32} & - a_{22} & \lambda \mu^{-1} a_{12} \\
\lambda^{-2} a_{41} & - \lambda^{-1} \mu^{-1} a_{31} & \lambda^{-1} \mu a_{21} & - a_{11}
\end{pmatrix}
$$
I claim that
\begin{align*}
\C[V]^T
= 
\C[&a_{11}, a_{22}, a_{12} a_{21}, a_{13} a_{31}, a_{14} a_{41}, a_{23} a_{32}, \\
	&a_{12} a_{23} a_{31}, a_{12} a_{13} a_{41}, a_{13} a_{21} a_{32}, a_{14} a_{21} a_{31}, \\
	&a_{12}^2 a_{23} a_{41}, a_{13}^2 a_{32} a_{41}, a_{14} a_{23} a_{31}^2, a_{14} a_{21}^2 a_{32}]
\end{align*}
This can almost be proved in the same way as for type $A_2$, though it's tricky to prove all the resulting $i_k$s are non-negative.
\textcolor{red}{Can we do this an easier way, e.g., by using Zariski dense?}

\section*{Computing the invariant ring for general $G$}
\textcolor{red}{To talk about: I think this is related to roots summing to zero.}

\section*{Appendix: Computing invariant rings}
\subsection*{$\C^\times$ acting on $\C \oplus \C$}
We'll compute $\mathbb{C}[V]^G$ for the example of $G = \mathbb{C}^\times$ acting on $V = \mathbb{C} \oplus \mathbb{C}$.
Note that 
$$\mathbb{C}[V]^G = \{f \in \mathbb{C}[V] \, : \, g \cdot f = f \text{ for all } g \in G\}$$
 where the action of $G$ on polynomials is given by
$$(g \cdot f)(v) = f(g^{-1} \cdot v).$$
We claim $\mathbb{C}[V]^G = \mathbb{C}[xy]$.
If $p \in \mathbb{C}[xy] \subseteq \mathbb{C}[x, y]$, there's a one-variable polynomial $\tilde p$ such that $p(x, y) = \tilde p (xy)$ for all $(x, y) \in V$.
Then for all $g \in G$,
$$(g \cdot p)(x, y) = p(g^{-1} \cdot (x, y)) = p(g^{-1} x, g y) = \tilde p((g^{-1} x)(g y)) = \tilde p(xy) = p(x, y).$$
Now let $p \in \C[V]^G$.
In general, $p \in \C[x, y]$ is of the form 
$$p(x, y) = \sum_{i, j=0}^d a_{ij} x^i y^j.$$
We also have
$$(g \cdot p)(x, y) = p(g^{-1} x, g y) = \sum_{i, j=0}^d a_{ij} (g^{-1} x)^i (g y)^j = \sum_{i, j=0}^d a_{ij} g^{j - i} x^i y^j.$$
Since $p$ is invariant, it holds for all $g \in G$ that $g \cdot p - p = 0$, i.e.,
$$\sum_{i, j=0}^d (g^{j-i} - 1)a_{ij} x^i y^j = 0.$$
In particular, $(g^{j-i} - 1) a_{ij} = 0$ for all $g \in G$ and $0 \le i, j \le d$.
When $i \ne j$, substituting $g \ne 1$ tells us we must have $a_{ij} = 0$.
We conclude that in each term of $p$, $x$ and $y$ have the same degree, implying $p \in \C[xy]$.

\subsection*{The natural action of $SL_2$ (\cite[Exercise 5]{KP96})}
Let $G = SL_2$ act on $V = \C^2$ by multiplication.
Note that this action only has two orbits, $\{0\}$ and $\C^2 \setminus \{0\}$.
Since the invariant ring can equivalently be characterised as the polynomials on $V$ which are constant on orbits, $\C[V]^G = \C$.
Indeed, if a polynomial is constant on $\C^2\setminus \{0\}$, it must be constant on $\C^2$ by continuity.

\subsection*{$SL_n$ acting on $M_n$ (\cite[\S 1.2]{KP96})}
Let $G = SL_n$ act on $V = M_n$ by left multiplication, i.e., if $g \in SL_n$ and $A \in M_n$, $g \cdot A = gA$.

We claim that $\C[V]^G = \C[\det]$.
It is clear that $\det$ is invariant (since $\det(gA) = \det(g) \det(A) = \det(A)$).
Now let $f \in \C[V]^G$.
Define $p \in \C[t]$ by
$$p(t) = f\left( 
\begin{psmallmatrix}
	t & & & \\
	   & 1 & & \\
	   & & \ddots & \\
	   & & & 1
\end{psmallmatrix}
\right).$$
Noting that any $A \in GL_n$ can be written
$$A = A
\begin{psmallmatrix}
	(\det A)^{-1} & & & \\
	   & 1 & & \\
	   & & \ddots & \\
	   & & & 1
\end{psmallmatrix}
\begin{psmallmatrix}
	\det A & & & \\
	   & 1 & & \\
	   & & \ddots & \\
	   & & & 1
\end{psmallmatrix}
=
\tilde g
\begin{psmallmatrix}
	\det A & & & \\
	   & 1 & & \\
	   & & \ddots & \\
	   & & & 1
\end{psmallmatrix},$$
where $\tilde g = A 
\begin{psmallmatrix}
	(\det A)^{-1} & & & \\
	   & 1 & & \\
	   & & \ddots & \\
	   & & & 1
\end{psmallmatrix} \in SL_n$.
Then since $f$ is invariant,
$$f(A) = f \left(\tilde g \begin{psmallmatrix}
	\det A & & & \\
	   & 1 & & \\
	   & & \ddots & \\
	   & & & 1
\end{psmallmatrix} \right)
=f \left(\begin{psmallmatrix}
	\det A & & & \\
	   & 1 & & \\
	   & & \ddots & \\
	   & & & 1
\end{psmallmatrix} \right)
= p(\det A)$$
for all $A \in GL_n$.
Since $GL_n$ is Zariski-dense in $M_n$, this implies $f(A) = p(\det A)$ for all matrices $A$, and $f \in \C[\det]$.

\subsection*{$SL_2$ acting by conjugation on $\mathfrak{sl}_2$}
Let $G = SL_2$ act on $V = \mathfrak{sl}_2$ by conjugation, meaning if
$$
g=\begin{pmatrix} a & b \\ c & d \end{pmatrix} \in G, 
\qquad
v=\begin{pmatrix} x & y \\ z & -x \end{pmatrix} \in V,
$$
we have
$$
g \cdot v = g v g^{-1}.
$$
I think we should get $\C[V]^G = \C[x^2 + yz]$.

\subsection*{$T \le SL_2$ acting by conjugation on $\mathfrak{sl}_2$}
Let $T \le SL_2$ be the subgroup of diagonal matrices of $SL_2$, acting on $V = \mathfrak{sl}_2$ by conjugation.
That is, if
$$
g = \begin{pmatrix} \lambda & 0 \\ 0 & \lambda^{-1} \end{pmatrix} \in T,
\qquad
v = \begin{pmatrix} x & y \\ z & -x \end{pmatrix} \in V,
$$
we have
$$g \cdot v = g v g^{-1} =
\begin{pmatrix} x & \lambda^2 y \\ \lambda^{-2} z & -x \end{pmatrix}.$$
We claim that $\C[V]^T = \C[x, yz]$.
It is clear from the action that $\C[x, yz] \subseteq \C[V]^T$, since $x$ and $yz$ are invariant.
Let $p \in \C[V]^T$.
Then setting $x =1$, we must have
$$p(1, y, z) = p(g\cdot(1, y, z)) = p(1, \lambda^2 y, \lambda^{-2} z).$$
A similar argument to the first example $\C^\times \curvearrowright \C^2$ shows that in each term of $p(1, y, z)$, $y$ and $z$ must have the same degree.
Then since $T$ acts on the $x$ component trivially, we can see $p \in \C[x, yz]$.

\subsection*{Exercise 6 of \cite{KP96}}
Determine the invariant rings $\C[M_2(\C)]^U$ and $\C[M_2(\C)]^T$ under left multiplication by the subgroup $U$ of upper triangular unipotent matrices and the subgroup $T$ of diagonal matrices of $SL_2$.

\subsection*{Exercise 7 of \cite{KP96}}
Let $T_n \le GL_n$ be the subgroup of invertible diagonal matrices.
If we choose the standard basis in $V = \C^n$ and the dual basis in $\C^*$ we can identify the coordinate ring $\C[\C \oplus C^*]$ with $\C[x_1, \ldots, x_n, \zeta_1, \ldots, \zeta_n]$.
Show that $\C[\C \oplus \C^*]^{T_n} = \C[x_1 \zeta_1, x_2 \zeta_2, \ldots, x_n \zeta_n]$.
What happens if one replaces $T_n$ by the subgroup $T_n'$ of diagonal matrices with determinant 1?
(Hint: All monomials in $\C[x_1, \ldots, x_n, \zeta_1, \ldots, \zeta_n]$ are eigenvectors for $T_n$.)


\begin{bibdiv}
\begin{biblist}
\bib{KP96}{webpage}{
   author={Kraft, Hanspeter},
   author={Procesi, Claudio},
   title={Classical Invariant Theory, A Primer},
   date={1996},
   url={https://dmi.unibas.ch/fileadmin/user_upload/dmi/Personen/Kraft_Hanspeter/Classical_Invariant_Theory.pdf}
}
\end{biblist}
\end{bibdiv}

\end{document}