\documentclass[handout]{beamer}
\usetheme{metropolis}

\usepackage{tikz}
\usepackage{eucal}
\usepackage{wrapfig}
\usepackage{xcolor}
\usepackage{caption}
\usepackage{subcaption}
\usepackage{microtype}
\usepackage{float}

\theoremstyle{definition}
\newtheorem{noitalicstheorem}{Theorem}

\DeclareMathOperator{\GL}{GL}
\DeclareMathOperator{\M}{M}

\newcommand{\C}{\mathbb{C}}

\definecolor{mynavy}{RGB}{35, 55, 59}
\definecolor{myorange}{RGB}{235, 129, 27}
\definecolor{myteal}{RGB}{92, 188, 172}
\definecolor{mygreen}{RGB}{11, 176, 161}
\definecolor{myblue}{RGB}{130, 158, 117}
\metroset{block=fill}


\title{Affine Toric Varieties and Torus Quotients}
\subtitle{Honours Thesis Defence}
\author{Declan Fletcher}
\date{November 2024}

\begin{document}
\frame{\titlepage}

\begin{frame}
\frametitle{}
Let $V$ be a rational representation of an algebraic torus $T$.

\textbf{Main result}: the affine GIT quotient $V /\!\!/ T$ is a toric variety.

\begin{enumerate}
\pause
\item Algebraic sets
\begin{enumerate}
\item[--] Algebraic sets, their ideals, the Nullstellensatz
\item[--] Polynomial maps, direct products, open subsets
\end{enumerate}
\pause
\item Affine varieties
\begin{enumerate}
\item[--] Varieties, the maximal spectrum
\item[--] Morphisms, tangent spaces
\end{enumerate}
\pause
\item Convex geometry
\begin{enumerate}
\item[--] Polyhedral, rational and strongly convex cones
\end{enumerate}
\end{enumerate}
\end{frame}

\begin{frame}
\frametitle{}
\begin{enumerate}
\item[4.] Affine toric varieties
\begin{enumerate}
\item[--] Semigroup algebras, toric varieties, their points
\item[--] Faces and open affine subsets, the torus action, singularities
\end{enumerate}
\pause
\item[5.] Torus quotients as toric varieties
\begin{enumerate}
\item[--] Algebraic groups, the affine GIT quotient
\item[--] The invariant ring $k[V]^T$, the cone of $V / \!\! / T$, examples
\end{enumerate}
\end{enumerate}
\end{frame}

\end{document}