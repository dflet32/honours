\subsection{Dominant morphisms}
Given the correspondence between morphisms and algebra homomorphisms, it is natural to ask if properties of the homomorphism inform properties of the morphism, and vice versa.
One property of morphisms which is determined algebraically is called \emph{dominant}.
A morphism is dominant if it has dense image in its codomain.
The following proposition relates this to an algebraic property:

\begin{proposition}
Let $\alpha : A \to B$ be a homomorphism of affine $k$-algebras, and let 
$$\varphi: \Spec(B) \to \Spec(A)$$
be the corresponding morphism of affine varieties.
Then $\varphi$ is dominant if and only if $\alpha$ is injective.
\end{proposition}

This result will be important when studying toric varieties, as it will tell us the open subset isomorphic to a torus is in fact dense in the toric variety.