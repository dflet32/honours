%%%%%%%%%%%%%%%%%%%%%%%%%%%%%%%%%%%%%%%%%%%%%%%%%%%%%%%%%%%%%%%%%%%%%%%%%%%%%%%%
% Affine geometric invariant theory
%%%%%%%%%%%%%%%%%%%%%%%%%%%%%%%%%%%%%%%%%%%%%%%%%%%%%%%%%%%%%%%%%%%%%%%%%%%%%%%%
\newpage
\section{Affine geometric invariant theory}
Geometric invariant theory (GIT) was invented by David Mumford in the 1960s as a method of constructing moduli spaces --- spaces whose whose points represent isomorphism classes of a given algebro-geometric object. These are spaces of solutions to a geometric classification problems.Mumford's work built on ideas in Hilbert's study of classical invariant theory. Given an action of a group $G$ on an algebraic variety $X$, geometric invariant theory asks how we can construct a quotient of $X$ by $G$ with favourable properties, as in general, the set of orbits $X/G$ will not have the structure of an algebraic variety. In this section, we will study GIT to see how to construct quotients in algebraic geometry.

\begin{definition}[\cite{Brion09}]
    A $G$-variety is a variety $X$ with an action of an algebraic group $G$, such that the action
$$G \times X \to X, \qquad (g, x) \mapsto g \cdot x$$
    is also a morphism of varieties.
\end{definition}

\todo{examples.}

\begin{definition}[\cite{Brion09}]
    Let $G$ be an algebraic group and $X$ a $G$-variety. A \emph{geometric quotient} of $X$ by $G$ is a variety $Y$ and a morphism $\pi : X \to Y$ such that:
    \begin{enumerate}[label=(\roman*)]
        \item $\pi$ is surjective;
        \item the fibers of $\pi$ are exactly the $G$-orbits in $X$;
        \item A subset $U \subseteq Y$ is open if and only if $\pi^{-1}(U)$ is open;
        \item For any open subset $U \subset Y$, the comorphism $\pi^*$ gives an isomorphism $\CC[U] \cong \CC[\pi^{-1}(U)]^G$.
    \end{enumerate}
\end{definition}

In view of (i)--(iii) above, the variety $Y$ can be identified with the quotient space $X/G$ equipped with the usual quotient topology. Furthermore, (iv) uniquely  

\todo{examples, particularly, cases where the geometric quotient doesn't exist.}

\begin{theorem}[\cite{Brion09}]
    Let $G$ be a reductive algebraic group, and $X$ an affine $G$-variety. Then:
    \begin{enumerate}[label=(\roman*)]
    \item The subalgebra $\CC[X]^G \subseteq \CC[X]$ is finitely generated.
    \item Let $f_1, \ldots, f_n$ be generators of the algebra $\CC[X]^G$. Then the image of the morphism
    $$X \to \CC^n, \qquad x \mapsto (f_1(x), \ldots, f_n(x))$$
    is closed and independent of the choice of generators.
    \item Denote by
    $$\pi=\pi_X : X \to X \git G$$
    the surjective morphism defined by (ii). Then every $G$-invariant morphism $f : X \to Y$, where $Y$ is an affine variety, factors through a unique morphism $\phi: X \git G \to Y$.
    \item For any closed $G$-stable subset $Y \subseteq X$, the induced morphism $Y \git G \to X \git G$ is a closed immersion. In other words, the restriction of $\pi_X$ to $Y$ may be identified with $\phi_Y$. Moreover, given another closed $G$-stable subset $Y' \subseteq X$, we have $\pi_X(Y \cap Y') = \pi_X(Y) \cap \pi_X(Y').$
    \item Each fiber of $\pi_X$ contains a unique closed $G$-orbit.
    \item If $X$ is irreducible, then so is $X \git G$. If in addition $X$ is normal, then so is $X \git G$.
    \end{enumerate}
\end{theorem}

The above map $\pi$ is uniquely determined by the universal property (iii).
The variety $X \git G$ is called the \emph{geometric invariant theory quotient}, otherwise known as the \emph{categorical quotient}.
In view of (v), $X \git G$ can be thought of as the closed $G$-orbits of $X$.

\question{
Why is it useful to define objects by universal properties?
I roughly understand that in this case, the categorical quotient offers an alternative to the geometric quotient, which doesn't generally exists.
However, what is the motivation for defining objects by universal properties in general?
The abelianization of a group maybe be a useful example to think about. From MATH3303, we know that for a given group $G$, $G^{\text{ab}}$ is an abelian group with a surjective map homomorphism $\pi : G \to G^{\text{ab}}$ such that for any homomorphism $\varphi: G \to H$, where $H$ is an abelian group, there is a unique homomorphism $\tilde \varphi$ such that the following commutes:
}
\[
\begin{tikzcd} 
% https://q.uiver.app/#q=WzAsMyxbMCwwLCJHIl0sWzEsMSwiSCJdLFsxLDAsIkdee1xcdGV4dHthYn19Il0sWzAsMiwiXFxwaSJdLFsyLDEsIlxcdGlsZGVcXHZhcnBoaSJdLFswLDEsIlxcdmFycGhpIiwyXV0=
	G & {G^{\text{ab}}} \\
	& H
	\arrow["\pi", from=1-1, to=1-2]
	\arrow["\varphi"', from=1-1, to=2-2]
	\arrow["\tilde\varphi", from=1-2, to=2-2]
\end{tikzcd}
\]
\textcolor{red}{Then one shows that $G^\text{ab} = G/[G, G]$ is a solution to the universal problem.}