\subsection{Affine space and algebraic sets}
Let $k$ be an algebraically closed field.
Affine $n$-space over $k$, denoted $\An_k$ or simply $\An$, is the set of $n$-tuples of elements of $k$:
$$\An = \{(a_1, \dots, a_n) : a_i \in k\}.$$
Elements of $\An$ are called points, and if $P = (a_1, \dots, a_n)$ then the $a_i$ are called the coordinates of $P$.
Let $A = k[x_1, \dots, x_n]$ be the polynomial ring in $n$ variables over $k$, which we interpret as the ring of polynomial functions on $\An$ by defining 
$f(P) = f(a_1, \dots, a_n)$ for $P \in \An$ and $f \in A$.
This allows us to talk about the zeros of a polynomial $f$, namely 
$$Z(f) = \{P \in \An : f(P) = 0\}.$$
More generally, if $T \subset A$ is any set of polynomials, we define the zero set of $T$ as the points where all polynomials in $T$ vanish:
$$Z(T) = \{P \in \An : f(P) = 0 \text{ for all } f \in T\}.$$
Note that if $\mathfrak{a}$ is the ideal generated by $T$, then $Z(T) = Z(\mathfrak{a})$.
Hilbert's celebrated \emph{basis theorem} tells us that $A$ is a Noetherian ring, so any ideal is finitely generated.
Then $\mathfrak{a} = (f_1, \ldots, f_r)$ for some $f_i \in A$, and $Z(T) = Z(\mathfrak{a}) = Z(\{f_1, \ldots, f_r\}).$

\begin{definition}
A subset $Y \subseteq \An$ is called \emph{algebraic} if $Y$ is the set of common zeros of a set of polynomials, i.e., if there exists $T \subseteq A$ such that $Y = Z(T)$.
\end{definition}

\begin{example}
Any line in $\mathbb{A}^2$ is the set of zeros of the polynomial $ax + by - c$ for some $a, b, c \in k$.
Therefore lines are algebraic.
The parabola $Z(\{y - x^2\})$ is an algebraic set.
\end{example}

\begin{proposition}[{\cite[Prop 1.1]{Hartshorne77}}]
We have the following properties of algebraic sets:
\begin{enumerate}
\item
If $Y_1$ and $Y_2$ are algebraic sets , then $Y_1 \cup Y_2$ is an algebraic set.

\item
If $\{Y_i\}_{i \in I}$ is an arbitrary collection of algebraic sets, then $\bigcap_{i \in I} Y_i$ is algebraic.

\item
$\emptyset$ and $\An$ are algebraic sets.
\end{enumerate}
\end{proposition}
\begin{proof}
(1) Suppose $Y_1 = Z(T_1)$ and $Y_2 = Z(T_2)$ where $T_1, T_2 \subseteq A$.
We claim that $Y_1 \cup Y_2 = Z(T_1 T_2)$, where $T_1 T_2 = \{f g \, : \, f \in T_1, g \in T_2\}$.
If $P \in Y_1 \cup Y_2$, assume without loss of generality that $P \in Y_1$.
Then for all $f \in T_1$, $f(P) = 0$, and $(fg)(P) = 0$ for all $fg \in T_1 T_2$ so $P \in Z(T_1 T_2)$.
Let $P \in Z(T_1 T_2)$.
If $f(P) = 0$ for all $f \in T_1$, then $P \in Z(T_1) = Y_1$.
Otherwise, let $f \in T_1$ such that $f(P) \ne 0$.
Since $P \in Z(T_1 T_2)$, we have $(fg)(P) = 0$ which implies $g(P) = 0$ for all all $g \in T_2$.
This means $P \in Z(T_2) = Y_2$ and in either case $P \in Y_1 \cup Y_2$.

(2) Let $\{Y_i\}_{i \in I}$ be algebraic sets such that $Y_i = Z(T_i)$ for each $i$.
Then
$$\bigcap_{i \in I} Y_i = \bigcap_{i \in I} Z(T_i) = \{P \in \An \, : \, f(P) = 0 \text{ for all } f  \in T_i, \text{ for all } i \in I\} = Z\left( \bigcup_{i \in I} T_i\right).$$

(3) Note that $\emptyset = Z(\{1\})$ and $\An = Z(\{0\})$.
\end{proof}

\begin{example}
The lines $y - x = 0$ and $y + x = 0$ are algebraic, and their union is defined by $y^2 - x^2 = (y-x)(y+x) = 0$.
If $\{Y_a\}_{a \in k}$, where each $Y_a = Z(\{y=ax\})$, is the set of lines through the origin, then $\bigcap Y_a = Z \left(\bigcup Y_a\right) = \{(0, 0)\}$.
Therefore $\{(0, 0)\}$ is an algebraic set in $\mathbb{A}^2$. 
\end{example}

\begin{definition}[Zariski topology]
The Zariski topology is on $\An$ is defined by taking the open subsets to be the complements of the algebraic sets, i.e., the algebraic sets are the closed sets.
The previous proposition implies this satisfies the axioms for a topology.
\end{definition}

\begin{example}[The Zariski topology on $\mathbb{A}^1$]
Let $Z(T)$ be an algebraic set where $T \subseteq A = k[x]$.
Since $Z(T) = Z(\mathfrak{a})$ for $\mathfrak{a} = \langle T \rangle$, and every ideal in $k[x]$ is principal ($k[x]$ is a PID since it is a Euclidean domain), $Z(T) = Z(\{f\})$ for some $f \in A$.
So every algebraic set in $\mathbb{A}^1$ is the zero set of a single polynomial.
Further, since $k$ is algebraic closed, any nonconstant $f$ can be written $f = c(x - a_1) \dots (x-a_n)$ for some $c, a_i \in k$.
So $Z(T) = Z(\{f\}) = \{a_1, \ldots, a_n\}$, meaning we can conclude every algebraic set in $\mathbb{A}^1$ is simply a finite set of points (or $\emptyset$ or $\mathbb{A}^1$).
Correspondingly, the open sets are $\emptyset$, $\mathbb{A}^1$, and the complements of finite sets.
Note this implies the Zariski topology is not Hausdorff;
if we try to find open sets $U$ and $V$ that separate the points, say, $0$ and $1$, $U$ and $V$ will contain an infinite number of points and must necessarily intersect.
\end{example}

\begin{definition}[Irreducible set]
A nonempty subset $Y$ of a topological space $X$ is irreducible if it cannot be expressed as the union $Y = Y_1 \cup Y_2$ of two proper subsets which are each closed in $Y$.
The empty set is not considered to be irreducible.
\end{definition}

\begin{example}
$\mathbb{A}^1$ is irreducible because its only proper subsets are finite yet it is infinite ($k$ is necessarily infinite since it is algebraically closed).
The algebraic set $Z(y^2 - x^2)$ is reducible since $Z(y^2 - x^2) = Z(y - x) \cup Z(y +x)$.
\end{example}

\begin{definition}[Affine algebraic variety]
An affine algebraic variety (or simply affine variety) is an irreducible closed subset on $\An$ with the induced topology.
An open subset of an affine variety is a quasi-affine variety.
\end{definition}

Affine and quasi-affine varieties are the first objects of study in algebraic geometry.
However, the study of these objects rely deeply on the relationship between subsets of $\An$ and ideals in $A$, so we need to investigate this relationship further (this will help us give examples of affine varieties).
For a subset $Y \subseteq \An$, define the ideal of $Y$ in $A$ by
$$I(Y) = \{f \in A \, : \, f(P) = 0 \text{ for all } P \in Y\}.$$
Now we have the function $Z$ which maps subsets of $A$ to algebraic sets, and $I$ which maps subsets of $\An$ to ideals.

Establishing the connection between ideals and sets relies on Hilbert's famous \emph{Nullstellensatz}:

\begin{theorem}[Hilbert's Nullstellensatz]
Let $k$ be an algebraically closed field, let $\a$ be an ideal in $A = k[x_1, \dots, x_n]$, and let $f \in A$ be a polynomial which vanishes at all points of $Z(\a)$.
Then $f^r \in \a$ for some integer $r > 0$.
\end{theorem}

This implies the following:

\begin{proposition}[{\cite[Prop 1.2]{Hartshorne77}}]
We have the following propeties of $Z$ and $I$:
\begin{enumerate}[label=(\roman*)]
\item
If $T_1 \subseteq T_2$ are subsets of $A$, then $Z(T_1) \supseteq Z(T_2)$.

\item
If $Y_1 \subseteq Y_2$ are subsets of $\An$, then $I(Y_1) \supseteq I(Y_2)$.

\item
For any two subsets $Y_1, Y_2 \subseteq \An$, we have $I(Y_1 \cup Y_2) = I(Y_1) \cap I(Y_2)$.

\item
For any ideal $\a \subseteq A$, $I(Z(\a)) = \sqrt{\a}$, the radical of $\a$.

\item
For any subset $Y \subseteq \An$, $Z(I(Y)) = \bar Y$, the (Zariski) closure of $Y$.
\end{enumerate}
\end{proposition}
\begin{proof}
(i) If all polynomials in $T_2$ vanish at $P$, then in particular all polynomials in $T_1$ vanish at $P$.

(ii) If $f$ is polynomial which vanishes at all $P \in Y_2$, then in particular $f$ vanishes at all $P \in Y_1$.

(iii) to do: complete this.
\end{proof}

\begin{corollary}[{\cite[Coro 1.4]{Hartshorne77}}]
There is a one-to-one inclusion-reversing correspondence between algebraic sets in $\An$ and radical ideals (i.e., ideals which are equal to their own radical) in $A$, given by $Y \mapsto I(Y)$ and $\a \mapsto Z(\a)$.
Furthermore, an algebraic set is irreducible if and only if its ideal is prime.
\end{corollary}
\begin{proof}
The first sentence is a direct consequence of (i), (ii), (iv) and (v) in the previous proposition.
Suppose $Y$ is irreducible.
\end{proof}

\begin{example}
This allows us to see $y = x^2$ is irreducible.
\end{example}

\begin{definition}[Affine coordinate ring]
If $Y \subseteq \An$ is an affine algebraic set, we define the affine coordinate ring of $Y$, denoted $A(Y)$, to be $A/I(Y)$.
\end{definition}

\begin{example}
The condition that an algebraic variety be irreducible implies $A(Y)$ is an integral domain.
To do: add an example and a counter-example.
\end{example}

\begin{example}
Coordinate ring of a line is $k[x]$.
\end{example}
