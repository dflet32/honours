\subsection{Hilbert series for $\CC[\frakg]^T$}
We give the general definition of the Hilbert series following \cite[\S 1.2]{Mukai03}.
Let $R = \CC[X_1, \ldots, X_n]$ and suppose $G$ acts on $\AA^n$ linearly, so that we get the action $(g\cdot f)(X) = f(g^{-1} \cdot X)$ on $R$.
Any polynomial $f(X) = f(X_1, \ldots, X_n)$ can be written uniquely as a sum of homogenous polynomials,
$$f(X) = f_0 + f_1(X) + \ldots + f_d(X),$$
where $d$ is the degree of $f$ and $\deg f_i = i$.
Since the action is linear, $\deg (g\cdot f_i) = i$, and $f$ is invariant for the action of $G$ if and only if each summand $f_i$ is invariant.
Letting $R_d$ denote the subspace of homogeneous polynomials of degree $d$, we can decompose $R$ and $R^G$ as
$$R = \bigoplus_{d \ge 0} R_d, \qquad R^G = \bigoplus_{d \ge 0} R^G \cap R_d.$$
The \emph{Hilbert series} $\mathrm{Hil}(R^G, t)$ of $R^G$ is the generating function for the dimensions of each homogeneous component of $R^G$.
That is,
$$\mathrm{Hil}(R^G, t):= \sum_{d \ge 0} \dim (R^G \cap R_d) t^d \in \ZZ[[t]].$$
The Hilbert series is a useful invariant of $R^G$, measuring its ``size and shape.''

For an example, consider $G = \langle \sigma \, | \, \sigma^2 = 1\rangle$ acting on $R = \CC[X_1, X_2]$ by $\sigma \cdot X_i = - X_i.$
Then a homogeneous polynomial is invariant if and only if it has even degree.
In particular, $R^G$ is generated by $X_1^2, X_1 X_2$ and $X_2^2$.
The subspace $R^G \cap R_{2d}$ is spanned by monomials $(X_1^2)^{k_1} (X_1 X_2)^{k_2} (X_2^2)^{k_3}$ such that $k_1 + k_2 + k_3 = d$, and there are ${d + 3 - 1 \choose 3 - 1} = {d + 2 \choose 2}$ solutions for $(k_1, k_2, k_3)$.
So then
$$\mathrm{Hil}(R^G, t) = \sum_{d \ge 0} {d + 2 \choose 2} t^{2d}.$$