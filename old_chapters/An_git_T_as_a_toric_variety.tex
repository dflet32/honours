\subsection{$\AA^n \git T$ as a toric variety}
In this section, we will study how the GIT quotient of a torus acting on affine space has the structure of a toric variety.
We follow \cite[\S 12]{Dolgachev03}.

\textcolor{red}{This is the original version of showing $\mathbb{A}^N \git T$ is a toric variety, from Dolgachev's book. I have tried to rewrite further down.}
Let $T = (\CC^\times)^r$ act linearly on $\AA^n = \CC^n$ by
$$t \cdot (z_1, \ldots, z_n) = (t^{\mathbf{a}_1} z_1, \ldots, t^{\mathbf{a}_n} z_n),$$
where if $t = (t_1, \ldots, t_r)$ and $\mathbf{a}_j = (a_{1j}, \ldots, a_{rj}) \in \ZZ^r$,
$$t^{\mathbf{a}_j} := t_1^{a_{1j}} \cdots t_r^{a_{rj}}.$$
Note that any representation of $T$ can be written this way, after diagonalising the action.
For a vector $\mathbf{m} =(m_1, \ldots, m_n) \in (\ZZ_{\ge 0})^n$, let $Z^\mathbf{m}:=Z_1^{m_1} \cdots Z_n^{m_n}$.
If $F \in \CC[Z_1, \ldots, Z_n]$ is given by a finite sum $\sum_{\mathbf{m}} a_{\mathbf{m}} Z^\mathbf{m}$, for some coefficients $a_{\mathbf{m}} \in \CC$, $F \in \CC[Z_1, \ldots, Z_n]^T$ if and only if for all $t \in T$ and all $z =(z_1,\ldots,z_n)\in\AA^n$,
$$F(z) = \sum_{\mathbf{m}} a_{\mathbf{m}} z^\mathbf{m} = \sum_{\mathbf{m}} a_{\mathbf{m}} t^{m_1 \mathbf{a}_1 + \ldots + m_n \mathbf{a}_n} Z^\mathbf{m} = F(t \cdot z).$$
Then we see a polynomial $F \in \CC[Z_1, \ldots, Z_n]$ is invariant under the action of $T$ if and only if it is a linear combination of monomials $Z^\mathbf{m}$ satisfying
\begin{equation}\label{invsys}
	m_1 \mathbf{a}_1 + \ldots + m_n \mathbf{a}_n = A \cdot \mathbf{m} = 0.
\end{equation}
% \todo{More explanation required, though this is shown properly when we look at $\CC[\frakg]^T$.}
Here $A = (a_{ij})$ is the $r \times n$ matrix of the exponents of the characters that $T$ acts by.       % DO YOU WANT TO USE THE WORD CHARACTER?
Let $\mathcal{M}$ be the semigroup of vectors $\mathbf{m} \in (\ZZ_{\ge 0})^n$ satisfying equation \ref{invsys}.
Since these $\mathbf{m}$ are precisely the exponents of invariant monomials for the action of $T$ on $\AA^n$, we have the isomorphism
$$\CC[Z_1, \ldots, Z_n]^T = \CC[Z^\mathbf{m} : \mathbf{m}\in (\ZZ_{\ge 0})^n \text{ and } A \cdot \mathbf{m} = 0] \cong \CC[\mathcal{M}].$$

To see $\AA^n \git T$ has the structure of a toric variety, we need to find a lattice $N$ and a cone $\sigma$ in $N_\RR$ such that $\mathcal{M} \cong \sigma^\vee \cap M$.
To this end, let $\ZZ^n \to \ZZ^r$ be the map given by the matrix $A$, and let $M = N^*$ be be the kernel of this map.
Then $M$ is a free abelian group of rank $l = n - \mathrm{rank}(A)$.
Let $(\ZZ^n)^* \to N = M^*$ be the map given by restriction of linear functions to $M$.
Let $e_1^*, \ldots, e_n^*$ be the dual basis of the standard basis $e_1, \ldots, e_n$ for $\ZZ^n$, and let $\bar{e}_1^*, \ldots, \bar{e}_n^*$ be the image of these vectors in $M^*$.
We define $\sigma$ to be the convex cone in $N_\RR$ generated by the vectors $\bar{e}_1^*, \ldots, \bar{e}_n^*$, i.e.,
$$\sigma = \Span_{\RR_{\ge 0}}\{ \bar{e}_1^* , \ldots, \bar{e}_n^*\}.$$
\textcolor{red}{The problem with this approach is that it gives no description of $N$ in any useful way.
That is, if $M$ is a sublattice of $\ZZ^n$, how do we describe the dual lattice $N$ and the cone $\sigma$?}

\begin{lemma}
We have that $\sigma^\vee \cap M = \mathcal{M}$.
\end{lemma}
\begin{proof}
Note that $\mathcal{M} = M \cap (\mathbb{Z}_{\ge 0})^n$.
Then
\begin{align*}
	\mathcal{M} &= \{m \in M : \bar{e}_i^*(m) \ge 0 \text{ for all } i = 1, \ldots, n\} \\
			&= \{m \in M : n(m) \ge 0 \text{ for all } n \in \sigma\} \\
			&= \sigma^\vee \cap M.
\end{align*}
\end{proof}

\begin{corollary}
The GIT quotient $\AA^n \git T$ is isomorphic to the affine toric variety $U_\sigma$.
\end{corollary}
\begin{proof}
We already know $\CC[Z_1, \ldots, Z_n]^T \cong \CC[\calM]$.
The lemma implies $\CC[\calM] = \CC[\sigma^\vee \cap M]$, so that $\AA^n \git T = \Spec(\CC[Z_1, \ldots, Z_n]^T) \cong \Spec(\CC[\sigma^\vee \cap M]) = U_\sigma.$ 
\end{proof}

Explicitly, let $v_1, \ldots, v_l$ be a basis for $M$ and construct an $l \times n$ matrix $B=(b_{ij})$ whose rows are given by the vectors $v_1, \ldots, v_l$.
If $N$ is identified with $\ZZ^l$ using the dual basis $v_1^*, \ldots, v_l^*$, then
$$e_i^* = \sum_{j=1}^l b_{ji} v_j^*$$
for each $i=1, \ldots, n.$
Then, $\sigma$ is spanned in $\RR^l \cong N_\RR$ by the columns $B_i$ of $B$.
\textcolor{red}{
This is concrete description is useful, but it's not insightful since it isn't directly described in terms of the characters of $T$ that $T$ is acting on $\AA^n$ by. 
I also realised that the definition of dual lattice is different to my preferred definition.
My preferred definition is that for a lattice $N$, the dual lattice is $N^* := \Hom(N, \ZZ)$, where $\mathrm{Hom}$ is just homomorphisms of lattices, equivalently, abelian groups.
The definition being used in the explicit description above is that if $N$ is a lattice defined as a discrete subgroup of $\RR^n$, the $N^* := \{y \in \RR^n : (x, y) \in \ZZ\}.$
}

\begin{example}
\textcolor{red}{This example is an example of the explicit description used above, but it's not important for now.}
Let us compute the cones corresponding to the affine GIT quotients $\frakg \git T$ for $G = \GL_2, \GL_3, \GL_4$.
The goal is to able to compute this for a general $G = \GL_n$.
We will just consider the action of $T$ on the off-diagonal entries of the Lie algebra, since the torus acts trivially on the diagonal.
\begin{enumerate}
\item
$G = \GL_2$.
This case is trivial but we include it for completeness.
Then $T = (\CC^\times)^2$ acts on $\AA^2 \cong \left\{\begin{pmatrix} 0 & y \\ z & 0 \end{pmatrix} : y, z \in \CC\right\}$ by
$$(t_1, t_2) \cdot (y, z) = (t_1 t_2^{-1} y, t_1^{-1} t_2 z).$$
The matrix $A$ is
$$A = \begin{pmatrix} 1 & -1 \\ -1 & 1 \end{pmatrix}.$$
$M = \ker A$ has rank $2 - \rank A = 1$.
There is a single basis vector $(1, 1)$.
It follows that $\sigma = \Span_{\RR_{\ge 0}} \{ 1 \} \subseteq \RR$.

\item
$G = \GL_3$.
The torus $T = (\CC^\times)^3$ acts on 
$$\AA^6 \cong \left\{\begin{pmatrix} 0 & a_{12} & a_{13} \\ a_{21} & 0 & a_{23} \\ a_{31} & a_{32} & 0 \end{pmatrix} : a_{ij}\in \CC\right\}$$
by
\begin{align*}
	&\qquad (t_1, t_2, t_3) \cdot (a_{12}, a_{23}, a_{13}, a_{21}, a_{32}, a_{31}) \\
	&= (t_1 t_2^{-1} a_{12}, t_2 t_3^{-1} a_{23}, t_1 t_3^{-1} a_{13}, t_1^{-1} t_2 a_{21}, t_2^{-1} t_3 a_{32}, t_1^{-1} t_3 a_{31}).
\end{align*}
Then,
$$
A = 
\begin{pmatrix}
	1 & 0 & 1 & -1 & 0 & -1 \\
	-1 & 1 & 0 & 1 & -1 & 0 \\
	0 & -1 & -1 & 0 & 1 & 1
\end{pmatrix},
$$
which by row reduction is equivalent
$$\tilde A = 
\begin{pmatrix}
	1 & 0 & 1 & -1 & 0 & -1 \\
	0 & 1 & 1 & 0 & -1 & -1 \\
	0 & 0 & 0 & 0 & 0 & 0
\end{pmatrix}
$$
Observe that columns are the coefficients for the roots in terms of the usual simple roots.
$M = \ker A$ has rank $6 - \rank(A) = 4$.
A choice of basis for $M$ is
$$\calB =\left\{
\begin{pmatrix}1\\0\\0\\1\\0\\0\end{pmatrix},
\begin{pmatrix}0\\1\\0\\0\\1\\0\end{pmatrix},
\begin{pmatrix}1\\1\\0\\0\\0\\1\end{pmatrix},
\begin{pmatrix}0\\0\\1\\1\\1\\0\end{pmatrix}
\right\}.$$
Then,
$$B = 
\begin{pmatrix}
	1&0&0&1&0&0 \\
	0&1&0&0&1&0 \\
	1&1&0&0&0&1 \\
	0&0&1&1&1&0 
\end{pmatrix},$$
and
$$\sigma = \Span_{\RR_{\ge 0}} \left\{
\begin{pmatrix} 0 \\ 0 \\ 1 \\ 0 \end{pmatrix},
\begin{pmatrix} 0 \\ 0 \\ 0 \\ 1 \end{pmatrix},
\begin{pmatrix} 1 \\ 0 \\ 1 \\ 0 \end{pmatrix},
\begin{pmatrix} 1 \\ 0 \\ 0 \\ 1 \end{pmatrix},
\begin{pmatrix} 0 \\ 1 \\ 1 \\ 0 \end{pmatrix},
\begin{pmatrix} 0 \\ 1 \\ 0 \\ 1\end{pmatrix}
\right\}$$
\end{enumerate}
\end{example}

\subsection{$\AA^n \git T$ as a toric variety, again}
\textcolor{red}{The goal of this section is rewrite the description of $\AA^n \git T$ as a toric variety in terms the characters that $T$ is acting by.
The goal is that the cone corresponding to $\AA^n \git T$ should admit a description in terms of these characters.
For simplicity, we will focus on the case when the characters that $T$ acts by are the roots (i.e., we are focusing on the adjoint action case).}
Let $G$ be a connected complex reductive group, $T$ a maximal torus of $G$, $\frakg$ the Lie algebra of $G$ and $\Phi$ the root system of $G$ with respect to $T$.
Let $V := \bigoplus_{\alpha \in \Phi} \frakg$ and fix $g_\alpha \in \frakg$ so that $\{g_\alpha : \alpha \in \Phi\}$ is a basis of $V$.
The adjoint action of $G$ on $\frakg$ gives an action of $T$ on $V$, where if $v = \sum_{\alpha\in\Phi} v_\alpha g_\alpha$ for some $v_\alpha \in \CC$, and $t \in T$, then
\begin{equation}\label{vsaction}
t \cdot v = \sum_{\alpha \in \Phi} v_\alpha \alpha(t) g_\alpha.
\end{equation}
Let $n$ denote the size of the root system $\Phi$.
We identify $V$ with the affine space $\AA^n$ by sending a vector to its coordinates;
$$v = \sum_{\alpha \in \Phi} v_\alpha g_\alpha \mapsto (v_{\alpha})_{\alpha \in \Phi} \in \AA^n.$$
Thus, we have an action of $T$ on $\AA^n$, where if $x =(x_\alpha)_{\alpha\in\Phi} \in \AA^n$ and $t \in T$, then
\begin{equation}\label{asaction}
t \cdot x = (\alpha(t) x_\alpha)_{\alpha\in\Phi}.
\end{equation}
The action in equation \ref{asaction} is the action in equation \ref{vsaction} restated using our identification of $V$ with $\AA^n$.

Write $\CC[X_\alpha : \alpha \in \Phi]$ for the coordinate ring of $\AA^n$.
The polynomial $X_\alpha$ is defined by
$$X_\alpha(x) := x_\alpha, \, \text{where } x = (x_\alpha)_{\alpha \in \Phi} \in \AA^n.$$
We now want to describe the invariant ring $\CC[X_\alpha:\alpha\in\Phi]^T$ for the action given in equation \ref{asaction}.
Let us define some notation for polynomials in $\CC[X_\alpha : \alpha \in \Phi]$.
Let $(\ZZ_{\ge 0})^\Phi$ denote the ring of functions $\Phi \to \ZZ_{\ge 0}$, and for $\eta \in (\ZZ_{\ge 0})^\Phi$, write $\eta = (\eta_\alpha)_{\alpha \in \Phi}$, where $\eta_\alpha = \eta(\alpha)$.
An element $\eta \in (\ZZ_{\ge 0})^\Phi$ defines a monomial in $\CC[X_\alpha : \alpha \in \Phi]$, namely
$$X^\eta := \prod_{\alpha \in \Phi} X_\alpha^{\eta_\alpha}.$$
A polynomial $p \in \CC[X_\alpha : \alpha \in \Phi]$ can be written as
$$p = \sum_{\eta} p_\eta X^\eta,$$
where the sum is over all $\eta \in (\ZZ_{\ge 0})^\Phi$ and all but finitely many of the coefficents $p_\eta \in \CC$ are zero.
Recall that since $T$ acts on $\AA^n$, there is an induced action of $T$ on the coordinate ring;
if $t \in T$ and $p \in \CC[X_\alpha : \alpha \in \Phi]$, then $(t \cdot p)(x) := p(t^{-1} \cdot x)$.
Using the definition of the action in equation \ref{asaction}, we then see that $T$ acts on a monomial $X^\eta$ by
$$t \cdot X^\eta = \prod_{\alpha \in \Phi} (\alpha(t^{-1}) X_\alpha)^{\eta_\alpha} = \left(\sum_{\alpha \in \Phi} - \eta_\alpha \alpha(t)\right) X^\eta.$$
It follows that a polynomial $p \in \CC[X_\alpha : \alpha \in \Phi]$ is invariant for the action of $T$ if and only if 
$$\sum_{\eta} p_\eta X^\eta = p = t \cdot p = \sum_{\eta} p_\eta \left(\sum_{\alpha \in \Phi} - \eta_\alpha \alpha(t)\right) X^\eta$$
for all $t \in T$.
Since $X^\eta$ and $X^\mu$ are linearly independent for distinct $\eta$ and $\mu$, the above equality holds if and only if
$$X^\eta = \left(\sum_{\alpha \in \Phi} - \eta_\alpha \alpha(t)\right) X^\eta$$
for all $\eta$ such that $p_\eta \ne 0$.
Equivalently, $p$ is invariant for the action of $T$ if and only if 
$$\sum_{\alpha \in \Phi} \eta_\alpha \alpha = 0$$
for all $\eta$ such that $p_\eta \ne 0$.
The following lemma summarises our discussion:

\begin{lemma}
We have that
$$\CC[X_\alpha : \alpha \in \Phi]^T = \CC\left[X^\eta : \eta \in (\ZZ_{\ge 0})^\Phi \text{ and } \sum_{\alpha \in \Phi} \eta_\alpha \alpha = 0\right].$$
\end{lemma}
\begin{corollary}
Let $\calM$ be the semigroup
$$\calM := \left\{\eta \in (\ZZ_{\ge 0})^\Phi : \sum_{\alpha \in \Phi} \eta_\alpha \alpha = 0\right\}.$$
If $\CC[\calM]$ denotes the semigroup algebra of $\calM$, then
$$\CC[X_\alpha : \alpha \in \Phi]^T \cong \CC[\calM].$$
\end{corollary}

To show that $\AA^n \git T$ has the structure of an affine toric variety, we now need to find a lattice $N$ and a cone $\sigma$ in $N_\RR$ such that $\calM \cong S_\sigma$, for then 
$$\AA^n \git T \cong \Spec(\CC[\calM]) \cong \Spec(\CC[S_\sigma]) = U_\sigma.$$
To do this, we first need a lemma that describes the dual lattice to a sublattice.

\begin{lemma}\label{dualsublattice}
Let $N_1$ and $N_2$ be lattices such that $N_2 \le N_1$.
Letting $M_i$ denote the dual lattice to $N_i$, we have that
$$M_2 \cong M_1 / K,$$
where $K := \{f \in M_1 : f(n) = 0 \text{ for all } n \in N_2\}.$
\end{lemma}
\begin{proof}
Let $\phi : M_1 \to M_2$ be the restriction map $f \mapsto \left. f \right|_{N_2}$.
Then $K = \ker \phi$, and $M_1 / K \cong \im \phi$, so we just need to show $\phi$ is surjective.
Let $\{n_1, \ldots, n_r\}$ be a set of generators for $N_2$, which we extend to a set of generators for $N_2$, $\{n_1, \ldots, n_r, n_{r+1}, \ldots, n_s\}$.
If $f \in M_2$, we can extend $f$ to a map $\tilde f \in \Hom(N_1, \ZZ)$ by defining $\tilde f$ on the generators of $N_1$ as
$$\tilde f(n_i) := \begin{cases} f(n_i) & \text{if } i = 1, \ldots, r, \text{ i.e., if } n_i \in N_2, \\ 0 & \text{if } i = r+1, \ldots, s, \text{ i.e., if } n_i \notin N_2. \end{cases}$$
Then $\phi(\tilde f) = \tilde f \big|_{N_2} = f$ and $\phi$ is surjective.
\end{proof}

To define the lattice $N$ of $\AA^n \git T$, we first define $M$, the dual lattice to $N$.
Let 
$$\ZZ^\Phi := \Fun(\Phi, \ZZ)$$ 
be the lattice of integer-valued functions on $\Phi$.
Then $(\ZZ^\Phi)_\RR \cong \RR^\Phi := \Fun(\Phi, \RR)$.
We have a map
$$\varphi : \ZZ^\Phi \to X^*(T), \qquad \eta = (\eta_\alpha)_{\alpha\in\Phi} \mapsto \sum_{\alpha \in \Phi} \eta_\alpha \alpha.$$
We define $M$ to be the kernel of $\varphi$,
$$M := \ker \varphi.$$
Then $M$ is a free abelian group of rank $l = |\Phi| - \rank(X^*(T))$.
By Lemma \ref{dualsublattice}, $N := M^\vee \cong (\ZZ^\Phi)^\vee / K$, where $K = \{f \in (\ZZ^\Phi)^\vee : \left. f \right|_{M} = 0\}.$
If $f \in (\ZZ^\Phi)^\vee$, we denote the restriction of $f$ to $M$ by $\overline{f} \in M^\vee$.
Let $\{e_\alpha^*\}_{\alpha\in\Phi}$ be the set of indicator functions in $\ZZ^\Phi$, which form a set of free generators for $\ZZ^\Phi$.
We let $\{e_\alpha\}_{\alpha\in\Phi}$ be the dual basis for $(\ZZ^\Phi)^\vee$, i.e., $e_\alpha(e^*_\beta) = \delta_{\alpha,\beta}$.
The cone $\sigma \subseteq N_\RR$ is defined to be the cone in $N \cong (\ZZ^\Phi)^\vee / K$ generated by the set $\{\overline{e_\alpha}\}_{\alpha\in\Phi}$, i.e.,
$$\sigma := \Span_{\RR_{\ge 0}} \left\{\overline{e_\alpha} : \alpha \in \Phi  \right\}.$$

\begin{proposition}
We have the following isomorphism of semigroups:
$$\sigma^\vee \cap M \cong \calM.$$
\end{proposition}
\begin{proof}
We have that
\begin{align*}
	\sigma^\vee \cap M &= \left\{ \eta \in \ZZ^\Phi : \sum_{\alpha\in\Phi} \eta_\alpha \alpha = 0 \text{ and } f(\eta) \ge 0 \text{ for all } f \in \sigma\right\} \\
	&= \left\{ \eta \in \ZZ^\Phi : \sum_{\alpha\in\Phi} \eta_\alpha \alpha = 0 \text{ and } \overline{e_\alpha}(\eta)=\eta_\alpha \ge 0 \text{ for all } \alpha \in \Phi \right\} \\
	&\cong \left\{\eta \in (\ZZ_{\ge 0})^\Phi : \sum_{\alpha\in\Phi} \eta_\alpha \alpha = 0\right\} \\
	&= \calM,
\end{align*}
where we have used in the second equality that checking $f(\eta) \ge 0$ for all $f$ in the cone $\sigma$ is equivalent to checking for all $f$ that are generators for the cone.
\end{proof}

\begin{corollary}
The GIT quotient $\AA^n \git T$ is isomorphic to the affine toric variety $U_\sigma$.
\end{corollary}
