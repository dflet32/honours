We study the affine GIT quotient $\frakg \git T$, where $G$ is a connected complex algebraic group with Lie algebra $\frakg$ and maximal torus $T$.
The variety $\frakg \git T$ can be explicitly computed for $G = \GL_2(\CC)$ or $G = \GL_3(\CC)$, but the problem is difficult for arbitrary $G$.
For any torus acting on an affine space, the GIT quotient has the structure of a toric variety.
We investigate toric varieties to use the toric variety structure of $\frakg \git T$ to elucidate its properties.


\subsection{Classical invariant theory}
When a group $G$ acts on a vector space $V$, there is an induced action of $G$ on the space of polynomial functions on $V$.
Specifically, if $p \in \CC[V]$ and $g \in G$,
$$(g \cdot p)(v) := p(g^{-1} \cdot v), \text{ for all } v \in V.$$
The \emph{ring of invariants} is then defined by
$$\CC[V]^G := \{p \in \CC[V] : g \cdot p = p \text{ for all } g \in G\}.$$

An important problem in classical invariant theory is to determine whether an invariant ring $\CC[V]^G$ is finitely generated.
David Hilbert showed that when $G$ is $\SL_n(\CC)$ acting on $V$ by a representation, $\CC[V]^G$ is indeed finitely generated \cite{Hilbert90}.
In Hilbert's influential list of 23 problems posed in 1900, the $14^\text{th}$ asks whether certain polynomial algebras are finitely generated;
a special case of the problem is whether $\CC[V]^G$ is finitely generated when $G$ is an arbitrary algebraic group.
In 1960, Masayoshi Nagata found an example where $\CC[V]^G$ is \emph{not} finitely generated \cite{Nagata60}.
However, when $G$ is a \emph{reductive} algebraic group, $\CC[V]^G$ is indeed finitely generated \cite[Theorem 4.51]{Mukai03}.

\subsection{Geometric invariant theory}
In the 1960s, David Mumford built on Hilbert's work and developed the theory of \emph{geometric invariant theory} (GIT) \cite{Mumford65}.
Given an algebraic group $G$ acting on a variety $X$, the aim of GIT is to construct a quotient variety, i.e., a variety whose points are in bijection with the orbits $X/G$.
Unfortunately, in general $X/G$ does not have the structure of a variety.
In the case when $X$ is a complex affine variety and $G$ a complex algebraic group, Mumford defined the affine GIT quotient, or categorical quotient,
$$X \git G := \Spec(\CC[X]^G).\footnote{In this work, $\Spec$ denotes the maximal spectrum instead of the prime spectrum, i.e., we consider affine varieties instead of affine schemes. See \S \ref{varietiessection} for further discussion of our definition of a variety.}$$
This is a complex affine variety which has $\CC[X]^G$ as its coordinate ring---intuitively, the points of $X \git G$ are in bijection with the closed orbits of the action of $G$ on $X$.
For a $\CC$-algebra $R$ to be the coordinate ring of an affine variety, it is necessary that $R$ is finitely generated.
Thus Hilbert's work elucidating when $\CC[X]^G$ is finitely generated shows $X \git G$ is a variety when $G$ is reductive.\footnote{Note that $\Spec(\CC[X]^G)$ can still be studied as a scheme even if $\CC[X]^G$ is not finitely generated, however, the finitely generated case is more manageable.}

Let us consider an example of a GIT quotient which is well understood.
Let $G = \GL_n(\CC)$ and $X = \frakg = \frakg\frakl_n(\CC)$.
We can think of $\frakg$ as the affine space $\CC^{n\times n}$ by choosing a basis and identifying $x \in \frakg$ with its coordinates.
Then $G$ acts on $X$ by conjugation, i.e., for $g \in G$ and $x \in \frakg$,
$$g \cdot x := g x g^{-1}.$$
\emph{Chevalley's restriction theorem} \cite[\S 23]{Humphreys72} implies that 
$$\CC[\frakg]^G \cong \CC[\frakh]^W,$$
where $\frakh$ is the Cartan subalgebra of diagonal matrices and $W = S_n$ is the Weyl group of $G$.
Here $W$ acts on $\frakh$ by permuting the diagonal entries of a matrix.
The ring $\CC[\frakh]^W$ is well-understood;
by the \emph{fundamental theorem of symmetric polynomials}, it is a free polynomial algebra with $n$ generators.
Specifically, $\CC[\frakh]^W \cong \CC[Y_1, \ldots, Y_n].$
Then, we have that
$$\frakg \git G = \Spec(\CC[\frakg]^G) \cong \Spec(\CC[Y_1, \ldots, Y_n]) = \CC^n.$$
Note that $\frakg \git G$ can be computed more generally when $G$ is any connected complex reductive group acting on its Lie algebra by the adjoint action.
Chevalley's restriction theorem again implies that $\CC[\frakg]^G \cong \CC[\frakh]^W$, and the \emph{Chevalley-Shephard-Todd} theorem \cite[\S 3]{Humphreys90} says $\CC[\frakh]^W$ is a free polynomial algebra;
we can again conclude $\frakg\git G$ is an affine space.
The GIT quotient $\frakg \git G$ is called the \emph{adjoint quotient}.

\subsection{Goals of this project}
Given the adjoint quotient is well-understood, it is natural to ask about the quotient variety of $\frakg$ by subgroups of $G$.
One choice of such a subgroup is a \emph{maximal torus}, which is a subgroup isomorphic to $(\CC^\times)^r$ for some $r \in \ZZ_{> 0}$.
The goal of this project is to study the GIT quotient $\frakg \git T$.

To do this, we investigate the ring of invariants $\CC[\frakg]^T$.
For a general connected algebraic group $G$, it may be out of reach to compute $\CC[\frakg]^T$, however, we can still study features of $\frakg \git T$ such as its dimension and singularities.

The first steps in this project were to compute examples.
In the cases when $G = \GL_2(\CC)$ or $G = \GL_3(\CC)$, it is tractable to compute $\CC[\frakg]^T$ by hand.
However, even $G = \Sp_4(\CC)$ becomes difficult.

\subsection{Toric varieties}
Studying $\frakg \git T$ can be generalised to investigating $\AA^n \git T$, where $T$ is any algebraic torus acting an affine space $\AA^n$.
One can prove that $\AA^n \git T$ has the structure of an \emph{affine toric variety}.
These are affine varieties which are determined by a \emph{cone} in a vector space.
The convex geometry of the cone determines many properties of the toric variety.
Given the difficulty of directly computing $\CC[\frakg]^T$, computing the toric variety structure and appealing to the many facts known about toric varieties may make the problem more manageable.

\subsection{Directions for further research}
We list some aims for the remainder of this project:
\begin{enumerate}
\item
Write a complete introduction to affine GIT, including the related notions of stability and semi-stability;
using this understanding, compute the stable and semi-stable points for the adjoint action of $G$ on $\frakg$.

\item
Compute the cone of the affine toric variety $\frakg \git T$.
Use this information to determine features of $\frakg \git T$, such as its dimension and singularities.

\item
Compute the stable and semi-stable points for the action of $T$ on $\frakg$.

\item
Investigate projective toric varieties.
The \emph{projective GIT quotient} $\AA^n \git_\chi \, T$, where $\chi$ is a character of $T$, is an example of these.
\end{enumerate}

\subsection{Contents of this document}
This document contains our progress so far in studying $\frakg \git T$.

We begin in $\S 2$ by introducing affine varieties.
This chapter can be considered background material needed to understand the project.
The majority of the chapter deals with algebraic subsets of affine space, and investigates the connection between these sets and ideals in polynomial rings.
We conclude in \S \ref{varietiessection} by discussing abstract affine varieties, which are defined as the maximal spectrum of a certain kind of $k$-algebra;
this is necessary to define the affine GIT quotient.

In $\S 3$, we define the affine GIT quotient $X \git G$ and give examples.
In particular, we investigate $\frakg \git T$ in the cases when $G = \GL_2(\CC)$ and $G = \GL_3(\CC)$; in these cases we can compute the invariant ring explicitly.
We compute a basis for $\CC[\frakg]^T$ in the general case in \S 3.2.

The final chapter is an introduction to affine toric varieties.
The prerequisite convex geometry is discussed, and in particular, we give a self-contained proof of an important fact (Theorem \ref{dualitythm}) for the theory of toric varieties, which is stated without proof in many texts on toric varieties such as \cite{Fulton93}, \cite{CLS11}, and \cite{Oda88}.
After defining affine toric varieties in terms of their cones, we prove $\frakg \git T$ is an affine toric variety.
