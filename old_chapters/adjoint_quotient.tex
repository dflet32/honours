\newpage
\section{The adjoint quotient, $\frakg \git G$}
In this section, we will study Chevalley's restriction theorem, which elucidates the structure of $\CC[\frakg]^G$ and allow us to understand $\frakg \git G$.
To understand the statement, we require some preliminaries.
All facts in this section can be found in \cite{Humphreys72}.

\subsection{The Weyl group of $\frakg$}
Assume that $\frakg$ is a non-zero semisimple Lie algebra.
If $\frakh$ is a Cartan subalgebra and $\Phi$ is the corresponding root system, recall the root-space decomposition
$$\frakg = \frakh \oplus \bigoplus_{\alpha \in \Phi} \frakg_\alpha.$$
We have that if $x \in \frakg_\alpha$ for $\alpha \in \Phi$, $\ad_x$ is nilpotent.

Since $\frakg$ is semisimple, the Killing form restricted to $\frakh$ is non-degenerate, giving an identification between $\frakh$ and $\frakh^*$.
Specifically, $\phi \in \frakh^*$ corresponds to the unique $t_\alpha \in \frakh$ such that $\phi(h) = \kappa(t_\phi, h)$ for all $h \in \frakh$.
Since $\Phi$ is a spans $\frakh^*$, we have the corresponding spanning set for $\frakh$:
$$\Phi \subseteq \frakh^* \longleftrightarrow \{t_\alpha \, | \, \alpha \in \Phi\}$$
This identification induces an inner product on $\frakh^*$, given by
$$(\phi, \psi) = \kappa(t_\phi, t_\psi).$$

For any non-zero vector $\alpha \in \frakh^*$, we have a corresponding \emph{reflection}, $\sigma_\alpha$, defined by
$$\sigma_\alpha(\beta) = \beta - \frac{2 (\beta, \alpha)}{(\alpha, \alpha)} \alpha.$$
We note that this is a linear map sending $\alpha$ to $-\alpha$ and fixing pointwise the reflecting hyperplane $P_\alpha = \{\beta \in \frakh^*\, | \, (\alpha, \beta)=0\}.$

The \emph{Weyl group} of $\frakg$, $W$, is defined to be the group generated by all reflections $\sigma_\alpha$ with $\alpha \in \Phi$, i.e.,
$$W =\langle \sigma_\alpha \, | \, \alpha \in \Phi \rangle \subseteq GL(\frakh^*).$$
Since $\Phi$ is an abstract root system (see \cite[\S 9.2]{Humphreys72}), each reflection $\sigma_\alpha$ permutes $\Phi$.
Then $W$ is contained in the symmetric group on $\Phi$, implying it is finite.

\subsection{Inner automorphisms of $\frakg$}
Recall that an \emph{automorphism} of $\frakg$ is a Lie algebra isomorphism $\frakg \to \frakg$.

\begin{example}
If $\frakg \subseteq \mathfrak{gl}(V)$ and $a \in GL(V)$ such that $a \frakg a^{-1} = \frakg$, it is straightforward to check that the conjugation map $x \mapsto a x a^{-1}$ preserves the commutator, implying the map is an automorphism of $\frakg$.
\end{example}

If $x \in \frakg$ and $\ad_x$ is nilpotent, say $(\ad_x)^k = 0$, we can define $\exp(\ad_x)$ using the usual power series definition, which terminates by nilpotence:
$$\exp(\ad_x) = 1 + \ad_x + \frac{1}{2!} (\ad_x)^2 + \ldots + \frac{1}{(k-1)!} (\ad_x)^{k-1}.$$
It turns out that $\exp(\ad_x)$ will be an automorphism of $\frakg$, (see \cite[\S 2.3]{Humphreys72} for the computation).
We define 
$$\mathrm{Int}(\frakg) := \langle \exp(\ad_x) \, | \, x \in \frakg \text{ is nilpotent}\rangle,$$
the \emph{inner automorphisms}.
One can prove that the inner automorphisms form a normal subgroup of the group of automorphisms of $\frakg$ (again see \cite[\S 2.3]{Humphreys72}).

\begin{example}
Let $\frakg = \mathfrak{sl}_2(\CC)$, with standard basis $\{e, h, f\}$.
A key example of an inner automorphism, which will be important for us later, is
$$\sigma = \exp(\ad_e)\exp(\ad_{-f})\exp(\ad_e).$$
A straightforward computation shows that $\sigma$ acts on the basis by
$$\sigma(e) = -f, \quad \sigma(f) = -e, \quad \sigma(h) = -h.$$
Consider the usual matrix exponentials 
$$\exp(e) = 
\begin{pmatrix}
	1 & 1 \\
	0 & 1 
\end{pmatrix},
\quad
\exp(-f) =
\begin{pmatrix}
	1 & 0 \\
	-1 & 1
\end{pmatrix},$$
which converge since $e, f$ are nilpotent matrices.
If
$$s = \exp(e) \exp(-y) \exp(e) =
\begin{pmatrix}
	0 & -1 \\
	1 & 1
\end{pmatrix},$$
we get that
$$s e s^{-1} = -f, \quad s f s^{-1} = -e, \quad s h e^{-1}=h,$$
so conjugation by $s$ acts in the same way as $\sigma$.
The general phenomena is:
if $\frakg \subseteq \mathfrak{gl}(V)$ and $x \in \frakg$ is nilpotent, then
$$\exp(x) y \exp(x)^{-1} = \exp(\ad_x)(y).$$
This means that if $\exp(\ad_x)$ is a generator for $\mathrm{Int}(\frakg)$, $\exp(x) \in G$ (here $G$ is the algebraic group that has $\frakg$ as it's Lie algebra) and then $\exp(\ad_x) \in \mathrm{Ad}(G)$.
So $\mathrm{Int}(\frakg) \subseteq \mathrm{Ad}(G)$.
%%%%%%%%%%%%
\textcolor{red}{Question: in Humphrey's book the statement $\CC[\frakg]^G \cong \CC[\frakh]^W$ takes $G = \mathrm{Int}(\frakg)$, but for our purposes we want $G$ to be an algebraic group acting on $\frakg$ by the adjoint action.
Then one hopes these are the same thing, i.e., $\mathrm{Int}(\frakg) = \mathrm{Ad}(G)$ as a subgroup of $\mathrm{Aut}(\frakg)$.
Is it clear why this is this case, or do we need some assumptions on $G$ for this to hold? 
I think it's enough if $G$ is generated by elements $\exp(x)$ for $x \in \frakg$ nilpotent.}
\end{example}

It is necessary for Chevalley's restriction theorem to realise elements of the Weyl group as inner automorphisms.
(Generally, an automorphism of $\Phi$ determines an automorphism of $\frakh$, which can be extended to $\frakg$.)
We want to find an inner automorphism $\tau_\alpha$ that restricted to $\frakh$ acts like the reflection $\sigma_\alpha$.
For any non-zero $x_\alpha \in \frakg_\alpha$ for $\alpha \in \Delta$, let $y_\alpha \in h_\alpha$ satisfy $[x_\alpha, y_\alpha]=h_\alpha$.
We claim that 
$$\tau_\alpha = \exp(\ad_{x_\alpha})\exp(\ad_{-y_\alpha})\exp(\ad_{x_\alpha})$$
acts like $\sigma_\alpha$ on $\frakh$.
The previous example tells us that
$$\tau_\alpha(x_\alpha) = - y_\alpha, \quad \tau_\alpha(y_\alpha) = - x_\alpha, \quad \tau_\alpha(h_\alpha) = -h_\alpha.$$
We have $\frakh = \ker \alpha \oplus \mathrm{span} \{h_\alpha\}$.
If $h \in \ker \alpha$, 
$$\ad_{x_\alpha}(h) = [x_\alpha, h] = - \alpha(h) x_\alpha = 0,\quad \ad_{-y_\alpha}(h) = [-y_\alpha, h] = \alpha(h) y_\alpha =0,$$
so we readily see that $\tau_\alpha(h) = h$ for all $h \in \ker \alpha$.
This shows $\tau_\alpha$ agrees with the reflection $\sigma_\alpha$ on $\frakh$.
In this way, we realise generators (and therefore all elements) of the Weyl group as elements of $\mathrm{Int}(\frakg)$.
However, Humphreys \cite[\S 14.3]{Humphreys72} notes that the subgroup of $\mathrm{Int}(\frakg)$ generated by the $\tau_\alpha$, $\alpha \in \Delta$ may be bigger than the Weyl, so the Weyl group is not necessarily a subgroup of $\mathrm{Int}(\frakg)$.

\subsection{The symmetric algebra}
Let $V$ be any vector space over a field $k$.
The \emph{symmetric algebra on $V$}, $S(V)$, is defined as
$$S(V) = T(V) /\langle x \oplus y - y \oplus x \, | \, x, y \in V\rangle.$$
$S(V)$ enjoys the following universal property:
if $A$ is a commutative algebra, for any linear map $f: V \to A$, there exists a unique algebra homomorphism $g: S(V) \to A$ such that the following diagram commutes:
% https://q.uiver.app/#q=WzAsMyxbMCwwLCJWIl0sWzEsMCwiUyhWKSJdLFsxLDEsIkEiXSxbMCwxLCJcXGlvdGEiXSxbMSwyLCJnIl0sWzAsMiwiZiIsMl1d
\[\begin{tikzcd}
	V & {S(V)} \\
	& A
	\arrow["\iota", from=1-1, to=1-2]
	\arrow["g", from=1-2, to=2-2]
	\arrow["f"', from=1-1, to=2-2]
\end{tikzcd}\]
If $B$ is a basis for $V$, then $S(V)$ can be identified with $k[B]$ through a canonical isomorphism.
In this way, $S(V)$ can be thought of as a ``coordinate-free'' polynomial ring.

In stating Chevalley's restriction theorem, we will consider $S(\frakg^*)$ to be algebra of polynomial functions on $\frakg$.
If $G$ acts on $\frakg$, it acts on $S(\frakg^*)$ by
$$g \cdot (e_1 \otimes \ldots \otimes e_m) = (g\cdot e_1) \otimes \ldots \otimes (g \cdot e_m),$$
where $g \in G$ acts on $e_i \in V^*$ in the usual way that an action of a group on a vector space gives an action on the functions on that vector space:
$$(g\cdot e_i)(v) = e_i(g^{-1} \cdot v).$$

\subsection{Statement of the theorem}
Let $G = \mathrm{Int}(\frakg)$, which acts on $\frakg$.
Identify $\CC[\frakg]$ with $S(\frakg^*)$, which has a natural action of $G$.
The polynomials fixed by the $G$ action are called the $G$-invariant polynomials, which is denoted $\CC[\frakg]^G$; this space is identified with $S(\frakg^*)^G$.
Since we have realised the elements of $W$ as inner automorphisms, we have an action of $W$ on $\frakh$ and hence $\CC[\frakh] \cong S(\frakh^*)$.
Therefore, we can also also consider the $W$-invariant polynomials on $\frakh$, $\CC[\frakh]^W \cong S(\frakh^*)^W$.

Any polynomial function on $\frakg$ restricts to a polynomial function on $\frakh$ (this is clear if we extend a basis of $\frakh$ to a basis of $\frakg$, and think of polynomial functions on $\frakg$ as polynomials in the dual basis).
If $p \in \CC[\frakg]^G$, in particular, $p$ is invariant under the action of the inner automorphisms $\tau_\alpha$, $\alpha \in \Delta$.
As $\left. \tau_\alpha \right|_\frakh$ is the reflection $\sigma_\alpha$, $\left. p \right|_\frakh$ is invariant under the generators of the Weyl group, so $\left. p \right|_\frakh \in \CC[\frakh]^W$.
This implies that the restriction of polynomials map $\CC[\frakg] \to \CC[\frakh]$, $p \mapsto \left. p \right|_\frakh$, induces an algebra homomorphism
$$\theta: \CC[\frakg]^G \to \CC[\frakh]^W.$$
Chevalley's restriction theorem says that \emph{$\theta$ is an isomorphism}.

\begin{example}
Let's consider the case when $\frakg = \mathfrak{sl}_3(\CC)$.
To explicitly see $\CC[\frakg]^G$, we will use the fact that this ring of invariants is generated by the coefficents of the characteristic polynomial.
Noting that
$$
\begin{pmatrix}
	h_1 & a & b \\
	x & h_2 - h_1 & c \\
	y & z & - h_2
\end{pmatrix} \in \frakg$$
has characteristic polynomial
$$- \lambda^3 + (ax + by + cz + h_1^2 - h_1 h_2 + h_2^2) \lambda 
+( acy + ah_2x + bh_1y - bh_2y + bxz + ch_1z + h_1^2h_2 - h_1h_2^2),$$
under the restiction map $\CC[\frakg] \to \CC[\frakh]$, the generators of $\CC[\frakg]^G$ map to 
\begin{align*}
	(ax + by + cz + h_1^2 - h_1 h_2 + h_2^2) &\mapsto h_1^2 - h_1 h_2 + h_2^2, \\
	(acy + ah_2x + bh_1y - bh_2y + bxz + ch_1z + h_1^2h_2 - h_1h_2^2) &\mapsto h_1^2 h_2 - h_1 h_2^2.
\end{align*}
Chevalley's restriction theorem tells us that
$$\CC[\frakg]^G \cong \CC[\frakh]^W = \CC[h_1^2 - h_1 h_2 + h_2^2, h_1^2 h_2 - h_1 h_2^2].$$
%%%%%%%%%%
\textcolor{red}{
I'm not sure that this seems correct.
As I have written it in previous sections, the actions of the simple reflections should be $h_1 \mapsto - h_1$ and $h_2 \mapsto - h_2$.
I don't think $\CC[h_1^2 - h_1 h_2 + h_2^2, h_1^2 h_2 - h_1 h_2^2]$ is invariant under this action.
The Weyl group should be $S_3$, but I don't see how this looks like permutations.
}
\end{example}

\subsection{The adjoint quotient $\frakg\git G$}
The real goal of study Chevalley's restriction theorem is to understand $\frakg \git G$.
We have that 
$$\CC[\frakg]^G \cong \CC[\frakh]^W.$$
Another theorem, the Chevalley-Shephard-Todd theorem, implies that
$$\CC[\frakh]^W \cong \CC[x_1, \ldots, x_r],$$
where $r = \mathrm{rank}(T)$.
Therefore, 
$$\frakg \git G = \mathrm{Spec}(\CC[\frakg]^G) \cong \mathrm{Spec}(\CC[x_1, \ldots, x_r]) = \CC^r.$$

We can also use the isomorphism $\CC[\frakg]^G \cong \CC[\frakh]^W$ to investigate the stability of $v \in \frakg$.
One formulation of stability says that $v \in \frakg$ is unstable if and only if for all $p \in \CC[\frakg]^G$, $p(v) = p(0)$.
%%%%%%%%%%
\textcolor{red}{
Ideally, we would be able to use the isomorphism to say that $v \in \frakg$ is unstable if and only if $\tilde p(0) = \tilde p(\left. v \right|_\frakh)$ for all $\tilde p \in \CC[\frakh]^W$, though I'm not sure how to get this to work.
If $v$ is unstable, for any $p \in \CC[\frakg]^G$, $p(0) = p(v)$, so under the restriction map $p \mapsto \tilde p = \left. p \right|_\frakh$, we will get $\tilde p (0) = \tilde p (\left. v \right|_\frakh).$
Conversely, for any $\tilde p \in \CC[\frakh]^W$, there is a unique $p \in \CC[\frakg]^G$ such that $\left. p \right|_\frakh = \tilde p$.
Suppose that for all $\tilde p \in \CC[\frakh]^W$, $\tilde p(0) = \tilde p(\left. v \right|_\frakh)$.
Does this imply $p(0) = p(v)$? 
This is not clear to me.
In fact, I don't think it's true.
Morally, we could have $v_1, v_2$ with the same restriction to the Cartan such that $v_1$ is unstable and $v_2$ is not, but $\tilde p(0) = \tilde p (\left. v_1 \right|_\frakh)$, $\tilde p(0) = \tilde p (\left. v_2 \right|_\frakh)$ for all $\tilde p \in \CC[\frakh]^W$.
Then $p(0) = p(v_1)$ but $p(0) \ne p(v_2)$ since $v_2$ is not stable.
How do we use Chevalley restriction to check stability?
}