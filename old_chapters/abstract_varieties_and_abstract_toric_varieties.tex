\newpage
\addtocontents{toc}{\vskip 1em}
\section{Abstract varieties}
In this section, we study abstract algebraic varieties --- these are spaces obtained by glueing together affine varieties.
An algebraic variety is the analogue in algebraic geometry of a manifold in differential geometry (these are spaces obtained by glueing open subsets of Euclidean space).
Our interest in algebraic varieties is motivated by the fact that there are toric varieties which are more general than the affine ones encountered in section \ref{affinetoricvarieties}, and these general toric varieties are in particular algebraic varieties. 
In the next section, we will studying general toric varieties using our understanding of algebraic varieties developed here.

In order to motivate the definition of algebraic varieties, we will first study projective varieties, which are an important special cases.
Then we will define general algebraic varieties, and see how they are determined by glueing affine varieties.





\subsection{Motivation: projective varieties}
Let $k$ be a field.
Projective $n$-space over $k$, denoted $\PP^n_k$ or $\PP^n$, is the quotient space
$$\PP^n_k := (k^{n+1} \setminus \{0\}) / \sim,$$
where $(a_0, \ldots, a_{n}) \sim (b_0, \ldots, b_{n})$ if $(a_0, \ldots, a_n) = \lambda (b_0, \ldots, b_n)$ for some $\lambda \in k \setminus \{0\}$.
This space can be viewed as the set of lines in $k^{n+1}$ which pass through $0$ in the following way:
if $\ell$ is a line in $k^{n+1}$ passing through the origin and $(a_0, \ldots, a_n)$ is any non-zero point in $\ell$, then the map
$$\ell \mapsto [(a_0, \ldots, a_n)]$$
defines a bijection between the set of lines and $\PP^n$.
The map is well-defined since $[(a_0, \ldots, a_n)] = [(b_0, \ldots, b_n)]$ if and only if $(a_0, \ldots, a_n) = \lambda (b_0, \ldots, b_n)$ for some $\lambda \in k \setminus \{0\}$, and it is a bijection since for any non-zero point in $k^{n+1}$, there is a unique line passing through the point and the origin.
The equivalence class $[(a_0, \ldots, a_n)]$ is usually denoted $(a_0 : \ldots : a_n)$.
If $P$ is the point $(a_0 : \ldots : a_n)$, the coordinates $(a_0 : \ldots : a_n)$ are called the homogeneous coordinates of $P$. 
Note that for any $P$, at least one homogeneous coordinate $a_i$ is non-zero.

In analogy with the affine theory, after defining projective space, we should define polynomials on the space so that algebraic subsets can be defined as the zero sets of polynomials.
However, the na\''ive approach of defining polynomials on $\PP^n$ to be polynomials in the homogeneous coordinates is not well-defined, since the homoegeneous coordinates are not unique;
we could then have that a polynomial takes different values on different homoegeneous coordinates for the same point.
The solution for defining algebraic subsets of $\PP^n$ is to use \emph{homogeneous} polynomials.

Example: elliptic curves.





\subsection{Abstract varieties}
Recall the definitions of a sheaf of $k$-algebras and ringed spaces from section \ref{affinetoricvarieties}.
The definition of algebraic prevariety extends the notion of an affine variety.
We follow Milne's definition of an algebraic variety \cite{Milne13}.

\begin{definition}
A topological space $V$ is called quasicompact if every open covering of $V$ has a finite subcovering.
\end{definition}

The condition to be quasicompact is often known as being compact.
The convention of Bourbaki is that a compact topological space is one that is quasicompact and Hausdorff \cite[\S 2 g.]{Milne13}; we follow this convention.

\begin{definition}
An algebraic prevariety over $k$ is a $k$-ringed space $(V, \calO_V)$ such that $V$ is quasicompact and every point of $V$ has an open neighbourhood $U$ for which $(U, \left. \calO_V \right|_U)$ is isomorphic to the ringed space of regular functions on an algebraic set over $k$.
\end{definition}

In other words, a ringed space $(V, \calO_V)$ is an algebraic prevariety over $k$ if there exists a finite open covering $V = \bigcup V_i$ such that $(V_i, \left. \calO_V \right|_{V_i})$ is an affine algebraic variety over $k$ for all $i$.

Recall that a topological manifold is required to be Hausdorff, a condition which excludes pathological topological behaviour like non-uniqueness of limits.
The analgous condition for prevarieties is called being separated.
An algebraic variety will then be a separated prevariety.

\begin{definition}
An algebraic prevariety $V$ is called separated if for every pair of regular maps $\varphi_1, \varphi_2 : Z \to V$, where $Z$ is an affine algebraic variety, the set $\{z \in Z : \varphi_1(z) = \varphi_2(z)\}$ is closed in $Z$.
\end{definition}

The following lemma tells us how we can obtain a prevariety by patching together ringed spaces:

\begin{proposition}
Suppose that the set $V$ is a finite union $V = \bigcup V_i$  of subsets $V_i$ and that each $V_i$ is equipped with a ringed space structure.
Assume that the following patching condition holds: for all $i,j$, $V_i \cap V_j$ is open in both $V_i$ and $V_j$ and $\left. \calO_{V_i} \right|_{V_i \cap V_j} = \left. \calO_{V_j} \right|_{V_i \cap V_j}$.
Then there is a unique ringed space structure on $V$ such that
\begin{enumerate}
\item
each inclusion $V_i \hookrightarrow V$ is a homeomorphism of $V_i$ onto an open set, and

\item
for each $i$, $\left. \calO_V \right|_{V_i} = \calO_{V_i}$.
\end{enumerate}
If every $V_i$ is an algebraic variety, then so also is $V$, and to give a regular map from $V$ to a prevariety $W$ amounts to giving a family of regular maps $\varphi_i : V_i \to W$ such that $\left. \varphi_i \right|_{V_i \cap V_j} = \left. \varphi_j \right|_{V_i \cap V_j}$.
\end{proposition}

Example: line with two origins

\subsection{The glueing construction}
Let $\{Y_i\}_{i\in I}$ be a finite set of affine varieties.
Suppose that for all $i, j \in I$, we have isomorphic open subsets $Y_{ij} \subseteq Y_i$ and $Y_{ji} \subseteq Y_j$.
Let $\{\phi_{ij} : Y_{ij} \to Y_{ji}\}_{i, j \in I}$ be isomorphism such that for all $i, j, k \in I$,
\begin{enumerate}
\item
$\phi_{ij} = \phi_{ji}^{-1}$

\item
$\phi_{ij}(Y_{ij} \cap Y_{ik}) = Y_{ji} \cap Y_jk$, and

\item
$\phi_{ik} = \phi_{jk} \circ \phi_{ij}$ on $Y_{ij} \cap Y_{ik}$.
\end{enumerate}
The data of the affine varieties $\{Y_i\}_{i\in I}$ and isomorphisms $\{\phi_{ij}\}_{i,j\in I}$ is called \emph{glueing data}.
The abstract variety $X$ which glues together the $\{Y_i\}_{i\in I}$ varieties is a certain topological space.
First, we construct the disjoint union of the varieties, $\hat X$, which is given by
$$\hat X := \bigsqcup_{i \in I} Y_i = \{(x, Y_i) : i \in I, x \in Y_i\}.$$
The set $\hat X$ is endowed with the disjoint union topology, where by definition, a set in $\hat X$ is open if it is a disjoint union of open subsets of the $Y_i$.
To construct $X$, we want to identify points in $\hat X$ if they belong to two isomorphic open subsets.
Specifically, define an equivalence relation on $\hat X$, $\sim$, by declaring $(x, Y_i) \sim (y, Y_j)$ if $x \in Y_{ij}$, $y\in Y_{ji}$ and $\phi_{ij}(x) = y$.
Condition (1) on the glueing isomorphisms ensures that $\equiv$ is reflexive and symmetric, and conditions (2) and (3) ensures it is transitive.
Now define the abstract variety $X$ to be $\hat X / \sim$ with the quotient topology;
this topology is called the Zariski topology on $X$.
For each $i \in I$, denote
$$U_i := \{[(x, Y_i)] \in X : x \in Y_i\}.$$
This is an open set of $X$, and the map $h_i : Y_i \to U_i$, $y \mapsto [(y, Y_i)]$ is a homeomorphism.
Then $X$ is locally isomorphic to an affine variety.

\begin{example}
For an example of glueing affine varieties, we consider how $\PP^1$ can be constructed by glueing together two copies of $\AA^1$.
Let $Y_1=\AA^1$, $Y_2 = \AA^1$ and $Y_{12} = k^\times \subseteq Y_1$, $Y_{21} = k^\times \subseteq Y_2$.
Define the glueing isomorphisms
$$\phi_{ij} : Y_{ij} \to Y_{ji}, \qquad t \mapsto t^{-1}.$$
It is clear that $\phi_{12} = \phi_{21}^{-1}$ so that axiom (1) for the glueing isomorphisms holds (axioms (2) and (3) are vacuously true).
Then, the variety obtained by glueing $Y_1$ and $Y_2$ is
$$X = Y_1 \sqcup Y_2 / \sim,$$
where $(x, Y_1) \sim (y, Y_2)$ if $x \ne 0$ and $y = x^{-1}$.
To see that $X$ is $\PP^1$, we can  think of the open sets
$$U_1 = \{[(a, Y_1)] : x \in Y_1\}, \qquad U_2 = \{[(b, Y_2)] : b \in Y_2\}$$
as the usual affine charts for $\PP^1$;
these are
$$U_x := \{(x:y) : x \ne 0\}, \qquad U_y := \{(x, y) : y \ne 0\},$$
and the maps
$$U_x \to U_1, \, (x : y)\mapsto [(x/y, Y_1)], \qquad \qquad U_y \to U_2, \, (x : y)\mapsto [(y/x, Y_2)]$$
are homeomorphisms.
Observe that if $(x : y) \in U_x \cap U_y$, the image of $(x : y)$ in $X$ of the above two maps are points points which are identified.
\end{example}

Example: line with two origins










\newpage
\addtocontents{toc}{\vskip 1em}
\section{Abstract toric varieties}
In this section, we present the general definition of an abstract toric variety.
While affine toric varieties correspond to cones in a vector space, these abstract toric varieties correspond to a collection of cones which ``fit together'' in a nice way --- these collections of cones are called fans.





\subsection{Fans}
Given a certain collection of cones called a fan, one constructs an abstract toric variety by glueing affine toric varieties $U_\sigma$ for cones $\sigma$ in the fan.
We now define a fan, and demonstrate how it encodes the glueing data.
As in chapter \ref{affinetoricvarieties}, let $N$ be a lattice, $M$ the dual lattice, and $N_\RR$ and $M_\RR$ the corresponding vector spaces.

\begin{definition}[{\cite[\S 1.4]{Fulton93}}]
A fan $\Delta$ in $N$ is a set of rational strongly convex polyhedral cones $\sigma$ in $N_\RR$ such that
\begin{enumerate}
\item
each face of a cone in $\Delta$ is also a cone in $\Delta$, and
\item
the intersection of two cones in $\Delta$ is a face of each.
\end{enumerate}
\end{definition}

For simplicity, we will assume that fans only contain a finite number of cones.
The idea of constructing the toric variety corresponding to $\Delta$ is that we consider each cone $\sigma$, and glue together the affine toric varieties $U_\sigma$.
The following lemma shows us that if $\tau$ is a face of a cone $\sigma$, then $U_\tau$ is an open subset of $U_\sigma$;
this will allow us to glue the affine toric varieties corresponding to cones in a fan.

Recall that a homomorphism of semigroups $S \to S'$ induces an algebra homomorphism $\CC[S] \to \CC[S']$ and hence a morphism $\Spec(\CC[S]) \to \Spec(\CC[S'])$.
In particular, if $\tau$ is contained in $\sigma$, there is an inclusion $\sigma^\vee \hookrightarrow \tau^\vee$ which determines a morphism $U_\tau \to U_\sigma$.

\begin{proposition}[{\cite[\S 1.3]{Fulton93}}]\label{faceembedding}
If $\tau$ is a face of $\sigma$, then the map $U_\tau \to U_\sigma$ embeds $U_\tau$ as a principal open subset of $U_\sigma$.
\end{proposition}

We need the following lemma from convex geometry:

\begin{lemma}[{\cite[\S 1.2]{Fulton93}}]
Let $\sigma$ be a rational convex polyhedral cone, and suppose $u \in S_\sigma$.
Then $\tau = \sigma \cap u^\perp$ is rational convex polyhedral cone. 
All faces of $\sigma$ have this form, and
$$S_\tau = S_\sigma + \ZZ_{\ge 0} \cdot (-u).$$
\end{lemma}
\begin{proof}
\todo{This is Fulton's proof, which is sparse on details. It would be good to write this clearer.}
If $\tau$ is a face, then $\tau = \sigma \cap u^\perp$ for any $u$ in the relative interior of $\sigma^\vee \cap \tau^\perp$, and $u$ can be taken to be in $M$ since $\sigma^\vee \cap \tau^\perp$ is rational.
Given $w\in S_\tau$, then $w + p \cdot u \in \sigma^\vee$ for large positive $p$, and taking $p$ to be an integer shows that $w \in S_\sigma + \ZZ_{\ge 0} \cdot (-u).$
\end{proof}

\begin{proof}[Proof of Proposition \ref{faceembedding}]
The lemma yields $u \in S_\sigma$ such that $\tau=\sigma\cap u^\perp$ and
$$S_\tau = S_\sigma + \ZZ_{\ge 0} \cdot (-u).$$
Then any basis element in $\CC[S_\tau]$ can be written as $\chi^{w - p u} = \chi^w / (\chi^u)^p$ for some $w \in S_\sigma$ and $p \in \ZZ_{\ge 0}$.
Then, $\CC[S_\tau] = \CC[S_\sigma]_{\chi^u}$, which is the algebraic version of the assertion.
\end{proof}





\subsection{Abstract toric varieties}
Let $\Delta$ be a fan in $N$.

\begin{definition}
The toric variety $X(\Delta)$ is constructed by glueing the affine toric varieties $U_\sigma$, as $\sigma$ ranges over all elements of $\Delta$.
For cones $\sigma, \tau \in \Delta$, the intersection $\sigma \cap \tau$ is a face of each and hence $U_{\sigma \cap \tau}$ is isomorphic to a principal open subset of each $U_\sigma$ and $U_\tau$, and these varieties can be glued along this open subvariety.
\end{definition}

The fact that the glueing is compatible follows from the order-preserving nature of the correspondence between cones and affine varieties.
\todo{Understand this better. Can we give explicit glueing maps $\phi_{ij}$ as we used in the definition of glueing?}

Example: one-dim toric varieties

\begin{proposition}
Let $\sigma$ and $\sigma'$ be cones such that $\tau = \sigma \cap \sigma'$ is a face of both.
Then,
$$S_\tau = S_\sigma + S_{\sigma'}.$$
\end{proposition}
\begin{proof}
It follows from the definitions that $S_\tau \supseteq S_\sigma + S_{\sigma'}$.
By Lemma \ref{}, there exists $u \in \sigma^\vee \cap (-\sigma')^\vee$ such that $\tau = \sigma \cap u^\perp = \sigma' \cap u^\perp$.
By Lemma \ref{}, we have $S_\tau = S_\sigma + \ZZ_{\ge 0} \cdot (-u)$.
Since $-u \in S_{\sigma'}$, we see $S_\tau \subseteq S_\sigma + S_{\sigma'}.$
\end{proof}



\subsection{The dense torus}





\subsection{The orbit-cone correspondence}





\subsection{Classification of toric surfaces}





\subsection{Toric morphisms}
