\documentclass{beamer}
\usetheme{metropolis}

\usepackage{tikz}
\usepackage{eucal}
\usepackage{wrapfig}
\usepackage{xcolor}
\usepackage{caption}
\usepackage{subcaption}
\usepackage{microtype}
\usepackage{float}

\theoremstyle{definition}
\newtheorem{noitalicstheorem}{Theorem}

\DeclareMathOperator{\GL}{GL}
\DeclareMathOperator{\M}{M}

\newcommand{\C}{\mathbb{C}}

\definecolor{mynavy}{RGB}{35, 55, 59}
\definecolor{myorange}{RGB}{235, 129, 27}
\definecolor{myteal}{RGB}{92, 188, 172}
\definecolor{mygreen}{RGB}{11, 176, 161}
\definecolor{myblue}{RGB}{130, 158, 117}
\metroset{block=fill}


\title{GIT and Affine Toric Varieties}
\subtitle{Mid-year Review}
\author{Declan Fletcher}
\date{2 August 2024}

\begin{document}
\frame{\titlepage}

\begin{frame}
\frametitle{Affine GIT}
\pause
Suppose a group $G$ acts on a complex affine variety $X = \mathrm{Spec}(R)$.
The $G$-invariant functions on $X$ are
$$R^G := \{f \in R : f(g \cdot P) = f(P) \text{ for all } g \in G, P \in X\}.$$
The GIT quotient is
$$X /\!\!/ G := \mathrm{Spec}(R^G).$$

\pause
Example: $\mathbb{C}^\times$ acting on $\mathbb{C}^2=\mathrm{Spec}(\mathbb{C}[X, Y])$ by
$$g \cdot (x, y) = (g x, g^{-1} y), 
\qquad g \in \mathbb{C}^\times, \, (x, y) \in \mathbb{C}^2.$$
Then,
$$\mathbb{C}[X, Y]^{\mathbb{C}^\times} = \mathbb{C}[XY], 
\quad \text{and} \quad \mathbb{C}^2 /\!\!/ \mathbb{C}^\times \cong \mathbb{C}.$$
\end{frame}

\begin{frame}
\frametitle{GIT quotient of a Lie algebra by a torus}
\pause
Let $G$ be a connected complex reductive group with Lie algebra $\mathfrak{g}$ and maximal torus $T$.
The adjoint action of $G$ on $\mathfrak{g}$ induces an action of $T$ on $\mathfrak{g}$.
The goal of this project is to study
$$\mathfrak{g} /\!\!/ T = \mathrm{Spec}(\mathbb{C}[\mathfrak{g}]^T).$$

\pause
Example: $G = \mathrm{GL}_2(\mathbb{C})$, $T$ is the subgroup of diagonal matrices, and $\mathfrak{g}=\mathfrak{gl}_2(\mathbb{C})$.
Then $T$ acts on $\mathfrak{g}$ by
$$\begin{pmatrix} a & 0 \\ 0 & b \end{pmatrix} \cdot \begin{pmatrix} x & y \\ z & w \end{pmatrix}
= \begin{pmatrix} x & a b^{-1} y \\ a^{-1} b z & w \end{pmatrix},$$
and therefore
$$\mathbb{C}[\mathfrak{g}]^T = \mathbb{C}[X, W, YZ], \quad \text{and} \quad \mathfrak{g} /\!\!/ T \cong \mathbb{C}^3.$$

\end{frame}

\begin{frame}
\frametitle{Affine toric varieties}
The quotient $\mathfrak{g} /\!\!/ T$ has the structure of an affine toric variety; \\
these are varieties determined by a cone in a vector space.

\pause
\begin{figure}[H]
    \centering
    \begin{tikzpicture}[scale=1.3]
        % Left diagram
        \begin{scope}[shift={(-2,0)}]
            % Shaded area
            \fill[gray!30] (0,0) -- (-1,2) -- (2,2) -- (2,0) -- cycle;
            
            % Axes
            \draw[->] (-1.25,0) -- (2.25,0);
            \draw[->] (0,-0.25) -- (0,2.25);
            
            % Lattice points
            \foreach \x in {-1,-0.5,0,0.5,1,1.5,2}
                \foreach \y in {0,0.5,1,1.5,2}
                    \fill (\x,\y) circle (1pt);

            % Vectors
            \draw[->, ultra thick] (0,0) -- (0.5,0);
            \draw[->, ultra thick] (0,0) -- (-0.5,1);
            \node at (1.75, 1.75) {$\sigma$};
        \end{scope}

        % Right diagram
        \begin{scope}[shift={(2,0)}]
            % Shaded area
            \fill[gray!30] (0,0) -- (0,2) -- (2,2) -- (2, 1) -- cycle;
            
            % Axes
            \draw[->] (-1.25,0) -- (2.25,0);
            \draw[->] (0,-0.25) -- (0,2.25);
            
            % Lattice points
            \foreach \x in {-1,-0.5,0,0.5,1,1.5,2}
                \foreach \y in {0,0.5,1,1.5,2} {
                    \fill (\x,\y) circle (1pt);
                }
                
			  % Labels
			  \node[anchor=south west] at (0, 0) {\!\!\! \tiny \alert{$1$}};
			  
			  %\node[anchor=south] at (-0.5, 0) {\tiny \alert{$X^{-1}$}};
			  %\node[anchor=south] at (-0.5, 0.5) {\tiny \alert{$X^{-1} Y$}};
			  %\node[anchor=south] at (-0.5, 1) {\tiny \alert{$X^{-1} Y^2$}};
			  %\node[anchor=south] at (-0.5, 1.5) {\tiny \alert{$X^{-1} Y^3$}};
			  
			  \node[anchor=south west] at (0, 0.5) {\!\!\! \tiny \alert{$Y$}};
			  \node[anchor=south west] at (0, 1) {\!\!\! \tiny \alert{$Y^2$}};
			  %\node[anchor=south] at (0, 1.5) {\tiny \alert{$Y^3$}};
			  \node[anchor=south west] at (0.5, 0) {\!\!\!\! \tiny \alert{$X$}};
			  \node[anchor=south west] at (0.5, 0.5) {\!\!\!\! \tiny \alert{$X Y$}};
			  \node[anchor=south west] at (0.5, 1) {\!\!\!\!  \tiny \alert{$X Y^2$}};
			  %\node[anchor=south] at (0.5, 1.5) {\tiny \alert{$X Y^3$}};
			  \node[anchor=south west] at (1, 0) {\!\!\!\! \tiny \alert{$X^2$}};
			  \node[anchor=south west] at (1, 0.5) {\!\!\!\! \tiny \alert{$X^2 Y$}};
			  \node[anchor=south west] at (1, 1) {\!\!\!\! \tiny \alert{$X^2 Y^2$}};
			  %\node[anchor=south] at (1, 1.5) {\tiny \alert{$X^2 Y^3$}};

            % Vectors
            %\draw[->, ultra thick] (0,0) -- (0,0.5);
            %\draw[->, ultra thick] (0,0) -- (1,0.5);
            \node at (1.75, 1.75) {\small{$\sigma^\vee$}};
        \end{scope}
    \end{tikzpicture}
\end{figure}
\noindent
$$U_\sigma = \mathrm{Spec}(\mathbb{C}[Y, XY, X^2 Y]) \cong \mathrm{Spec}(\mathbb{C}[U, V, W]/(UW - V^2)).$$

\pause
Properties of the variety are informed by properties of the cone.
\end{frame}

\begin{frame}
\frametitle{What I have done so far}
\begin{enumerate}
\pause \item 
Computed examples of $\mathfrak{g} /\!\!/ T$ when $G = \mathrm{GL}_2(\mathbb{C})$ and $G = \mathrm{GL}_3(\mathbb{C})$.

\pause \item 
Computed a basis for the invariant ring $\mathbb{C}[\mathfrak{g}]^T$ for general $G$:
$$\mathbb{C}[X^\eta Y^\mu : \eta \in \mathcal{A}, \mu \in (\mathbb{Z}_{\ge 0})^r],$$
where
$$\mathcal{A} := \left\{\eta \in (\mathbb{Z}_{\ge 0})^\Phi : \sum_{\alpha \in \Phi} \eta_\alpha \alpha = 0\right\}.$$

\pause \item
Proved that $\mathfrak{g} /\!\!/ T$ has the structure of a toric variety.
\end{enumerate}
\end{frame}

\begin{frame}
\frametitle{Goals for the remainder of the project}
\begin{enumerate}
\pause \item
Compute the cone corresponding to the affine toric variety $\mathfrak{g} /\!\!/ T$, and use this to study features of the variety.

\pause \item
Investigate the concepts of stability and semi-stability in GIT, and compute the stable and semi-stable points for the action of $T$ on $\mathfrak{g}$.

\pause \item
Learn about the projective GIT quotient and projective toric varieties;
the projective GIT quotient $\mathfrak{g} /\!\!/_\chi T$, where $\chi$ is a character of $T$, is a projective toric variety.
\end{enumerate}
\end{frame}

\end{document}